\chapter{Killing vectors and Killing tensors}\label{Killingchapter}
To obtain a simple form of the geodesic equations of a spacetime, a useful tool are Killing vectors. In a simple way, the Killing vectors are objects that inform us of symmetries of the spacetime and its metric. Using their properties we can obtain first integrals of the geodesic equations. Therefore, understanding the Killing vectors of a space time is an important step to obtain more simple equations for the geodesic flow.

\section{Properties of Killing vectors}

Formally Killing vectors are defined as:
\begin{equation}
\mathcal{L}_\xi g = 0,
\end{equation}
where $\mathcal{L}_\xi$ is the Lie derivative along the vector field $\xi$. If the manifold has a torsion-free metric connection $\nabla$, this expression becomes:
\begin{align}\label{killingformula}
{L}_\xi g_{a b} &= \xi^c \nabla_c g_{a b} + g_{a c} \nabla_b \xi^c + g_{c b} \nabla_a \xi^c \\ \nonumber
&= \nabla_b \xi_a +  \nabla_a \xi_b = 0 ,
\end{align}
where $g_{\alpha \beta}$ is the spacetime metric and the indices will be raised and lowered with $g$.

\section{Killing vectors and Lagrangian symmetries}

We will see that Lagrangian symmetries correspond to the Killing vectors of the spacetime. For this purpose consider:
\begin{align}\label{lagrangianodeff}
&v \in T_p M \\
&L(v,p)= \frac{1}{2} g|_p (v|_p,v|_p)
\end{align}
Considering a generic vector $Y \in T_p M$, we will move $\mathcal{L}$ along the one-parameter group $\phi_t$ generated by $Y$. If we name $\phi_t$ the tangent aplication $\phi$ we will have that:
\begin{align}
Y(L(v|_p,p))&= \frac{d}{dt}|_{t=0} \left( L (\phi_{*_t}(v),\phi_t(p)) \right)  \\
&=\frac{d}{dt}|_{t=0} \left( \frac{1}{2} g|_{\phi_t(p)} (\phi_{*_t}(v),\phi_{*_t}(v)) \right) \label{Lagrangiansimetrie}
\end{align}
In geometric terms all you are doing is moving the vector $ v $ by the group of diffeomorphisms $\phi_t$ and contracting it with the metric evaluated at $\phi_t(p)$. This process is the same as applying the pull-back to the metric evaluated at $\phi_t(p)$ and contract it in $p$ with the vector being evaluated at $p$. That is, given $\omega \in T^*_pM$, $v \in T_pM$ and $\phi_t$ a diffeomorphism, they fulfill:
\begin{equation}
\phi_t^*(\omega|_{\phi_t(p)})|_p(v|_p)= \omega ( \phi_{*_t}(v|_p))|_{\phi_t(p)}
\end{equation}
Therefore, from \cref{Lagrangiansimetrie} it follows that:
\begin{align}
 & \frac{1}{2} \frac{d}{dt}|_{t=0} \phi^*_t(g|_{\phi_t(p)})|_p (v|_p,v|_p) \\
 &=-\frac{1}{2} \lim_{t \to 0} \frac{ \phi^*_0(g|_{\phi_t(p)}) - \phi^*_t(g|_{\phi_t(p)})|_p }{t}(v|_p,v|_p) =-\frac{1}{2} \mathcal{L}_Y(g)(v|_p,v|_p)
\end{align}
If $\mathcal{L}_Y(g)=0$ then $Y$ is a Killing field and is also a symmetry of the Lagrangian.

\section{Killing vectors and first integrals}

Killing vectors are useful among many other reasons for defining conserved quantities. The definition of the conserved quantities is simply the dot product of the Killing vector for the tangent vector of the geodesic:
\begin{align}
 u^\beta \nabla_\beta (\xi_\alpha u^\alpha)&= u^\beta \xi_\alpha \nabla_\beta  u^\alpha+ u^\beta u^\alpha \nabla_\beta \xi_\alpha  \\
&= u^\beta u^\alpha \nabla_\beta \xi_\alpha  = u^\alpha u^\beta \nabla_\beta  \xi_\alpha  \\ &= \frac{1}{2} \left( u^\alpha u^\beta \nabla_\beta \xi_\alpha  + u^\alpha u^\beta \nabla_\alpha \xi_\beta  \right) = 0 \label{killingintegrals}
\end{align}
and hence we obtain that, along the geodesic:
\begin{equation}\label{killingconstants}
g(u,\xi) = cte
\end{equation}

\section{Noether currents and Killing vectors}

It is convenient to relate the Noether currents defining the symmetries of a Lagrangian and the Killing vectors of the metric. Noether currents of a Lagrangian satisfied that:
\begin{equation}
\dot{q^\alpha} \nabla_\alpha J + \ddot{q^\alpha} \nabla_\alpha J= 0
\end{equation}
where $q^\alpha$ and $ \dot{q}^\alpha$ are the are natual coordinates on the tangent bundle $J$ is defined, in case that the Lagrangian remains invariant under a symmetry generated by $\xi$, as:
\begin{equation}
J = \frac{\partial \mathcal{L}}{\partial (\dot{q^\alpha})} \xi^\alpha
\end{equation}
At infinitesimal level, the transformation whose generator is $ \xi $ takes the form:
\begin{equation}
q^\alpha \to q^\alpha + \xi^\alpha
\end{equation}
In the case of the Lagrangian  of \cref{lagrangianodeff}:
\begin{align}
J &= \frac{\partial \frac{1}{2} g_{\alpha \beta} \dot{q^{\alpha}} \dot{q^{\beta}} }{\partial (\dot{q^\gamma})} \xi^\gamma =\frac{1}{2} (g_{\gamma \beta} \dot{q^{\beta}} + g_{\alpha \gamma} \dot{q^{\alpha}}) \xi^\gamma \\ &= g_{\gamma \beta} \dot{q^{\beta}} \xi^\gamma =g(\dot{q},\xi)
\end{align}
And we get that for Killing vectors, the Noether currents are exactly the conserved geometric quantities associated Killing vector $\xi$.

\section{Killing Tensors}
A generalization of the Killing vector equation can be achieved generalizing the \cref{killingformula} for tensors:
\begin{equation}
\nabla_{(\alpha} T_{\beta \gamma )} = 0
\end{equation}
Thus, the following conserved quantities (using \cref{killingintegrals} ) will hold:
\begin{equation}
T_{\alpha \beta} u^\alpha u^\beta = cte
\end{equation}
There are Killing tensors that comes from tensor products of Killing vectors, and therefore does not provide independent conserved quantities. In other words, the Killing tensors:
\begin{equation}
T_{i j} = \xi_i \otimes \xi_j
\end{equation}
generate the conserved quantities:
\begin{align}
T_{i j} u^i u^j = c_i c_j
\end{align}
where $c_i=g(u,\xi_i)$.
\section{Killing algebra}
An useful property is that, given two Killing vectors $\xi_1$ y $\xi_2$:
\begin{equation}
\xi_3=[\xi_1,\xi_2]
\end{equation}
Is also a Killing vector because:
\begin{equation}
\mathcal{L}_{[\xi_1,\xi_2]} g=[\mathcal{L}_{\xi_1} ,\mathcal{L}_{\xi_2} ]g= 0
\end{equation} 
It is well-known that the set of all the complete vector fields on a manifold constitute also a Lie algebra which is naturally identifiable to the Lie algebra of the isometry group of the (semi-Riemannian) manifold.  When completeness is dropped, analogous considerations hold locally.