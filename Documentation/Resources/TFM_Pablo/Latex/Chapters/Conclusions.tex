%************************************************
\chapter{Conclusions}\label{ch:Concl}
%************************************************

The main result of this work is a detailed understanding of the motion
of particles in free fall in the Kerr geometry, both for massless and massive
particles. We have focused on the analysis of the axis of symmetry, the equatorial plane, and, in particular, the disk inside the singular ring of the Kerr black hole. Some of the most remarkable aspects in this work are, for example, the importance of the excluded regions, which provide the necessary topology to the physical region of the phase space to cover the whole structure of the
Penrose-Carter diagram in a single two-dimensional phase space. We have obtained completely new results on the geodesic flow in the axis of symmetry and in the inner disk from the point of view of dynamical systems, which as discussed above, is a novel approach to the problem. Some of these results are the existence of two critical points in the geodesic flow for timelike particles in the axis of symmetry, the fact that the axis of symmetry is stable, the result about the period of the lower energy orbit (which unexpectedly increases as the rotation increases) or the fact that only null geodesics that starts in the inner disk can remain in it. We have also found that the spacetime inside the singular ring behaves as the Minkowsky spacetime, analogous to the behavior of a thin, electrically charged shell, in which there is no electrical field. In the study of the movement along the axis of symmetry we have made a complete taxonomy of all kinds of movements as well as returning points and critical values of the kinetic parameters. Also, we have found some strange features as that the maximum time expended in travel along the Penrose diagram increases with the value of the rotation parameter $a$. To complete the understanding of the geodesic flow in the axis of symmetry and the equatorial plane we have obtained the necessary conditions that lead to future causal geodesic, which allow us to interpretate uniquely all curves in the phase portraits. The description of all geodesic flow along the maximal extension of the Kerr spacetime in terms of one two-dimensional phase space thanks to the future geodesic conditions is a great simplification to the geodesic analysis and is a very novel approach to this problem. Finally, we have derived an Energy-like equation for the exterior geodesic flow in the equatorial plane, which has a lot of advantages respect to the classical formulation of the geodesic equations.