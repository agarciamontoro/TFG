\chapter{Introduction to differential geometry}

\section{Differentiable manifolds}

Roughly speaking, a manifold is a topological space that, locally, looks like the Euclidean space $\R^n$. This similitude is essential, and will let us control the manifold as if we were working in the Euclidean space; generally, the definitions concerning manifolds and the properties proved from them will be based on the known properties of $\R^n$.

The following definition specifies the formal concept of a topological manifold:

\begin{definition}[N-dimensional topological manifold]
    Let $M^n$ be an $n$-dimensional topological space. The space $M^n$ is called a topological manifold if the following properties are satisfied:
    \begin{enumerate}
        \item $M^n$ is locally homeomorphic to $\R^n$. \label{def:manifold:homeo}
        \item $M^n$ is a Hausdorff space. \label{def:manifold:haussdorf}
        \item $M^n$ has a countable topological basis. \label{def:manifold:basis}
    \end{enumerate}
\end{definition}

The first property states that, for every point $p \in M^n$, there exists an open neighbourhood $U \subset M^n$ of $p$ and a homeomorphism
\[
    h \colon U \to V
\]

with $V \subset \R^n$ an open set.

One could think that the Hausdorff property is redundant, as the local homeomorphism may imply this topological characteristic. This is not true, and the usual counterexample is the line with two origins.

Let $M = \R \cup p$ be the union of the real line and a point $p \notin \R$. Define a topology in this space with $\R \subset M$ as an open set and the neighbourhoods of $p$ being the sets $(U \setminus \{0\}) \cup \{p\}$, where $U$ is a neighbourhood of $0 \in \R$. This space is locally Euclidean but not Hausdorff: the intersection of any two neighbourhoods of the points $0 \in \R$ and $p$ is non-empty.

\begin{figure}[bth]
    \myfloatalign
    \begin{tikzpicture}
      \draw[thick] (-5,0) -- (-0.05,0);
      \draw[very thick,<->] (-1,0) -- (-0.05,0);
      \draw[fill] (0,0) circle [radius=0.05];
      \node[below] at (0,0) {0};
      \draw[thick] (0.05,0) -- (5,0);
      \draw[very thick,<->] (0.05,0) -- (1,0);
      \node[right] at (5,0) {$\R$};

      \draw[fill,Maroon] (0,0.5) circle [radius=0.05];
      \node[right,Maroon] at (0,0.5) {$p$};
    \end{tikzpicture}
    \caption[Line with two origins]{Line with two origins.}
    \label{fig:2origin}
\end{figure}

The last property of the definition will prove to be key in our study, as it will let us define metrics on the manifold.

\subsection{Charts}

%TODO:0 Proofread this introduction. Particularly: ressemblance, Plato's world of Ideas, lowering, grasp.

The main characteristic of the manifolds, its ressemblance to the Euclidean space, have to be exploited in order to understand the nature of the mathematical object.

The conceptual space where the manifolds live can be thought as the Plato's world of Ideas, where everything is pure but cannot be understood without studying particular examples.

The idea of the manifold will be understood, then, taking pieces of the manifold and lowering them to the real word; \ie, the Euclidean space, where we will be able to \emph{physically} touch the manifold.

The essential tool to make this happen will be the coordinate charts. These tools are like prisms to see the manifold from the Euclidean perspective, and they will let us grasp the nature of the ideal concept of a manifold.

\begin{definition}[Coordinate chart]
    A \emph{coordinate chart} ---or \emph{coordinate system}--- in a topological manifold $M^n$ is a homeomorphism $h \colon U \to V$ from an open subset of the manifold $U \subset M$ onto an open subset of the Euclidean space $V \subset \R^n$.

    We call $U$ a \emph{coordinate neighbourhood} in $M$.
\end{definition}

One single chart may not cover all the manifold. In order to understand the whole manifold, we need a set of charts that describe it completely.

\begin{definition}[Coordinate atlas]
    Let
    \[
    A = \{h_\alpha \colon U_\alpha \to V_\alpha / \alpha \in I\}
    \]
    be a set of coordinate charts in a topological manifold $M^n$, where $I$ is a family of indices and the open subsets $U_\alpha \subset M$ are the corresponding coordinate neighbourhoods.

    $A$ is said to be an \emph{atlas} of M if every point is covered with a coordinate neighbourhood; \ie, if $\cup_{\alpha \in I} U_\alpha = M$.
\end{definition}

% Examples?

\subsection{Differentiable structures}

The concept of manifold is quite general and includes a vast set of examples. We can impose, however, some properties on the smoothness of the manifold to restrict the objects we will work with.

This section introduces the concept of differentiable structure, whose definition is key in the later description of differentiable manifolds, the core concept of this chapter.

The first question in this study is the following: a chart describe perfectly a single piece of the manifold, but what happens when the domains of a pair of charts overlap? The following two definitions precise the concepts involved in this question.

% TODO: Add the usual conmutative diagram for the transition maps

\begin{definition}[Transition map]
    Let $M^n$ be a manifold and $(U, \phi)$, $(V, \psi)$ a pair of coordinate charts in $M^n$ with overlapping domains
    \[
        U \cap V \neq \emptyset
    \]

    The homeomorphism between the open sets of the Euclidean space $\R^n$
    \[
        \psi \circ \phi^{-1} \colon \phi(U \cap V) \to \psi(U \cap V)
    \]
    is called a \emph{transition map}.
\end{definition}

\begin{definition}[Smooth overlap]
    Two charts $(U, \phi)$, $(V, \psi)$ are said to overlap smoothly if their domains are disjoint ---\ie, if $U \cap V  = \emptyset$--- or if the transition map $\psi \circ \phi^{-1}$ is a diffeomorphism.
\end{definition}

The description of two charts that overlap smoothly can be naturally extended to the concept of smooth atlas, that will make possible to do calculus on the manifold.

\begin{definition}[Smooth coordinate atlas]
    An atlas $A$ is said to be smooth if every pair of charts in $A$ overlap smoothly.
\end{definition}

But what happens if we define two different atlases in the manifold? Will the calculus depend on this choice? Fortunately we can find, for each manifold, one particular atlas that contain every other atlases defined there. It is formally described in the following definition and its uniqueness is proved in Proposition \autoref{prop:max-atlas-uniq}.

\begin{definition}[Complete atlas]
    A \emph{complete atlas} ---or \emph{maximal atlas}--- on $M^n$ is a smooth atlas that contains each coordinate chart in $M^n$ that overlaps smoothly with every coordinate chart in $M^n$.
\end{definition}

\begin{proposition}[Complete atlas uniqueness]
    Let $M$ be a topological manifold.

    \begin{itemize}
        \item Every smooth atlas on $M$ is contained in a complete atlas.
        \item Two smooth atlas on $M$ determine the same complete atlas if and only if its union is a smooth atlas.
    \end{itemize}
    \label{prop:max-atlas-uniq}
\end{proposition}

\begin{definition}[Differentiable structure]
\end{definition}

\begin{definition}[Differentiable manifold]
\end{definition}
