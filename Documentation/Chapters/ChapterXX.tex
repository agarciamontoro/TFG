\chapter{Introduction to differential geometry}

\section{Differentiable manifolds}

Roughly speaking, a manifold is a topological space that, locally, looks like the euclidean space $\Real^n$. This similitude is essential, and will let us control the manifold as if we were working in the euclidean space; generally, its properties will be proved using the known properties of $\Real^n$.

The following definition specifies the formal concept of a topological manifold:

\begin{definition}[N-dimensional topological manifold]
    Let $M^n$ be an $n$-dimensional topological space. The space $M^n$ is called a topological manifold if the following properties are satisfied:
    \begin{enumerate}
        \item $M^n$ is locally homeomorphic to $\Real^n$. \label{def:manifold:homeo}
        \item $M^n$ is a Hausdorff space. \label{def:manifold:haussdorf}
        \item $M^n$ has a countable topological basis. \label{def:manifold:basis}
    \end{enumerate}
\end{definition}

The first property states that, for every point $p \in M^n$, there exists an open neighbourhood $U \subset M^n$ of $p$ and a homeomorphism
\[
    h \colon U \to V
\]

with $V \subset \Real^n$ an open set.

The local homeomorphism does not imply the manifold to be Hausdorff, and this will be an essential property throughout the study of these spaces. The usual counterexample is the line with two origins: let $M = \Real \cup p$ be the union of the real line and a point $p \notin \Real$. Define a topology in this space with $\Real \subset M$ as an open set and the neighbourhoods of $p$ being the sets $(U \setminus \{0\}) \cup \{p\}$, where $U$ is a neighbourhood of $0 \in \Real$. This space is locally euclidean but not Hausdorff: the intersection of any two neighbourhoods of the points $0 \in \Real$ and $p$ is non-empty.

%TODO: add a figure with the two-origins line

The last property will prove to be key in our study, as it will let us define metrics on the manifold.

\subsection{Charts}

\begin{definition}[Coordinate chart]
\end{definition}

\begin{definition}[Coordinate atlas]
\end{definition}

% Examples?

\subsection{Differentiable structures}

% Transition maps. Smooth transition maps
\begin{definition}[Transition maps]
\end{definition}

\begin{definition}[Smooth coordinate atlas]
\end{definition}

\begin{definition}[Maximal atlas]
\end{definition}

\begin{proposition}[Maximal atlas uniqueness]
\end{proposition}

\begin{definition}[Differentiable structure]
\end{definition}

\begin{definition}[Differentiable manifold]
\end{definition}
