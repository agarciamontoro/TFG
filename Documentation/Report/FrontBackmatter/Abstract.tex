%*******************************************************
% Abstract
%*******************************************************
%\renewcommand{\abstractname}{Abstract}
\pdfbookmark[1]{Abstract}{Abstract}
\begingroup
\let\clearpage\relax
\let\cleardoublepage\relax
\let\cleardoublepage\relax

\chapter*{Abstract}
General relativity theory predicts the existence of black holes, interesting objects that curve the spacetime in such a way that even the light cannot escape its attraction. The simulation of the trajectories followed by light near these bodies is of great interest for the scientific community, and this work serves this need.

As an introduction to general relativity theory from a basic mathematical background, this work studies the basic concepts and results needed in order to build a ray tracer that generates images of what an observer near a black hole would see.

The implemented code is parallelized using \ac{GPGPU} techniques, obtaining a software that is up to 125 times faster than the non-parallelized solutions.

The implemented code is distributed under a free software license in order to let the community download, use and contribute to the code without restrictions.

\vfill

\endgroup

\vfill

\textsc{\textbf{Keywords}}: black hole, differential geometry, ray tracer, general relativity, geodesics, graphics processing units, Kerr spacetime, semi-Riemannian geometry, parallelization.
