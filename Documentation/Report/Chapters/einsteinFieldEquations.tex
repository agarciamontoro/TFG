\chapter{Einstein field equations}

\section{Einstein tensor of a metric}

\begin{definition}[Einstein tensor]
	Let $(M,g)$ be a four-dimensional spacetime. The symmetric 2-covariant tensor field
	\[
		G \defeq \operatorname{Ric}-\frac{1}{2}Sg,
	\]
	where Ric is the Ricci tensor of $g$ and $S$ the scalar curvature of $g$, is called the \emph{Einstein tensor of $g$}.
\end{definition}

It can be proved\fixme{Reference?} that $G$, the Einstein tensor of $g$, satisfies the following:

\begin{enumerate}
	\item $\operatorname{trace}_gG=-S$ and so $\mathrm{Ric}=G-(1/2)(\mathrm{trace}_gG)g$, where $\operatorname{trace}_gG$ denotes the contraction of the $(1,1)$-tensor field $g$-equivalent to $G$. Therefore, to have $G$ is equivalent to have $Ric$, and both tensors have the same physical information.
	\item If for any symmetric 2-covariant tensor field $T$, we define \[G_k:=T+k(\mathrm{trace}_gT)g,\] for a fixed $k\in \mathbb{R}$, then the map $T \longmapsto G_k$ is involutive if and only if $k=0$ or $k=-\frac{1}{2}$.
	\item $\mathrm{Ric}=\lambda\,g$ if and only if $G=-\lambda\,g$. A spacetime such that Ric is proportional to the metric tensor $g$, or equivalently, $G$ is proportional to $g$, is called an \emph{Einstein spacetime}.
\end{enumerate}

\section{Stress-energy tensors}

\begin{definition}[Stress-energy tensor]
	A \emph{stress-energy tensor} on a spacetime $M$ is a symmetric 2-covariant tensor field $T$ such that
	\begin{equation}
	\label{eq:stress}
		T(V,V) \geq 0
	\end{equation}
	for any timelike vector $V \in T_p M \; \forall p \in M$.
\end{definition}

Note that a continuity argument (see \autoref{sec:toporemarks}) shows that inequality \ref{eq:stress} also holds true if $V$ is a lightlike vector on $T_p M$.

It is commonly argued that the mathematical way to express that gravity attracts on average is
\begin{equation}
	\operatorname{Ric} (V,V) \geq 0
\end{equation}
for any timelike tangent vector $V \in T_p M$ and for all $p \in M$; that is, the Ricci tensor of a physically realistic spacetime must be an stress-energy tensor. This assumption restricts the familiy of Lorentzian metrics that can be physically significant.

\section{Einstein Field Equations}

Consider a stress-energy tensor $T$ on a spacetime $(M,g)$ such that
\begin{enumerate}
	\item The function \begin{equation}
		\label{eq:stressderived}
		T - \frac{1}{2}(\operatorname{trace}_g T) g
	\end{equation} is also a stress-energy tensor.
	\item $T$ satisfies the conservation law \begin{equation}
		\label{eq:div0}
		\operatorname{div} \widehat{T} = 0,
	\end{equation}
	where $\widehat{T}$ is the 2-contravariant tensor field $g$-equivalent to $T$.
\end{enumerate}

We say that the spacetime $(M,g)$ obeys the Einstein field equation with respect to $T$ if
\begin{equation}
	\label{eq:einstein}
	G(g) = T,
\end{equation}
where $G$ is the Einstein tensor of $g$, that is,
\begin{equation}
	\label{eq:einstein2}
	\Ric - \frac{1}{2}Sg = T.
\end{equation}

Note that we can also write, using previous formula \fixme{Which one?}, that
\begin{equation}
	\label{eq:einstein3}
	T - \frac{1}{2}(\operatorname{trace}_g T)g = \Ric.
\end{equation}

Therefore, assumption \ref{eq:stressderived} agrees with the timelike convergent condition, which express, using the Ricci tensor, that gravitational effects are on average attractive.

On the other hand, assumption \ref{eq:div0} is mandatory, as we know that the Einstein tensor $G$ satisfies
\[
	\operatorname{div} \hat{G} = 0.
\]

Einstein field equation (\autoref{eq:einstein} or \autoref{eq:einstein2}) postulated how matter and radiation in a region of the universe can be described by a Lorentzian metric $g$.

In fact, \autoref{eq:einstein} is similar to the Poisson equation,
\[
	\Delta\phi = k\rho,
\]
where $k>0$ is a universal constant, $\rho$ is the function describing the density of matter and $\phi$ is the potential function.

This equation postulated, in the pre-relativistic physics, how matter can be described by a potential function.

Note that if we know $\phi$, then the corresponding gravitational field is 
\[
	-\nabla \phi,
\]
where $\nabla$ denotes the usual gradient for functions on an open subset of the Euclidean space $\R^3$.

Then, the gravitational force, $F$, acting on a mass $m$, is obtained as follows:
\[
	F = -m\nabla\phi.
\]

We see then that, roughly speaking, $F$ had the same role in the old physics as the curvature tensor has now in Relativity.

Imagine now the trivial stress-energy tensor
\[
	T = 0.
\]

If $(M,g)$ obeys the Einstein field equation with respect to $T = 0$, then, using \autoref{eq:einstein3}, we realize that
\[
	Ric = 0.
\]

Assume now that $(M,g)$ obeys the Einstein field equation
\begin{equation}
	\label{eq:absence}
	Ric = 0.
\end{equation}

Then, it is trivial that $T=0$.

We can conclude that the mathematical way of expressing the absence of matter and radiation on the spacetime is the one shown on \autoref{eq:absence}, which is called the \emph{vacuum Einstein field equation}.