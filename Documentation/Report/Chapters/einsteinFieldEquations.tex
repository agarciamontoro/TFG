\chapter{Einstein field equations}

\section{Einstein tensor of $g$}

Define

$$G:=\mathrm{Ric}-\frac{1}{2}S\,g$$where Ric
is the Ricci tensor of $g$ and $S$ the scalar curvature of $g$.
The symmetric 2-covariant tensor field $G$ is called the
Einstein tensor of $g$. It satisfies:

\vspace{3mm}

\noindent {\textbf{ (1)}} $\mathrm{trace}_gG=-S$ and
so $\mathrm{Ric}=G-(1/2)(\mathrm{trace}_gG)g$. Therefore, to have
$G$ is equivalent to have $Ric$, and both tensors have the same
physical information.

\vspace{3mm}

\noindent{\textbf{ (2)}} If for any symmetric
2-covariant tensor field $T$, we define
$G_k:=T+k(\mathrm{trace}_gT)g$, for a fixed $k\in \mathbb{R}$,
then the map $T \longmapsto G_k$ is involutive if and only if
$k=0$ or $k=-\frac{1}{2}$.

\vspace{3mm}

\noindent {\textbf{ (3)}} $\mathrm{Ric}=\lambda\,g$ if and only if
$G=-\lambda\,g$. A spacetime such that Ric is proportional to the
metric tensor $g$, or equivalently, $G$ is proportional to $g$, is
called Einstein.