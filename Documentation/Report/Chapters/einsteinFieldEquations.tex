\chapter{Einstein field equations}

\section{Einstein tensor of a metric}

For a 4-dimensional spacetime, define
\[
	G \defeq \operatorname{Ric}-\frac{1}{2}Sg,
\]
where Ric is the Ricci tensor of $g$ and $S$ the scalar curvature of $g$. The symmetric 2-covariant tensor field $G$ is called the Einstein tensor of $g$. It satisfies:

\begin{enumerate}
	\item $\operatorname{trace}_gG=-S$ and so $\mathrm{Ric}=G-(1/2)(\mathrm{trace}_gG)g$, where $\operatorname{trace}_gG$ denotes the contraction of the $(1,1)$-tensor field $g$-equivalent to $G$. Therefore, to have $G$ is equivalent to have $Ric$, and both tensors have the same physical information.
	\item If for any symmetric 2-covariant tensor field $T$, we define $G_k:=T+k(\mathrm{trace}_gT)g$, for a fixed $k\in \mathbb{R}$, then the map $T \longmapsto G_k$ is involutive if and only if $k=0$ or $k=-\frac{1}{2}$.
	\item $\mathrm{Ric}=\lambda\,g$ if and only if $G=-\lambda\,g$. A spacetime such that Ric is proportional to the metric tensor $g$, or equivalently, $G$ is proportional to $g$, is called Einstein.
\end{enumerate}

\section{Stress-energy tensors}

\begin{definition}[Stress-energy tensor]
	A \emph{stress-energy tensor} on a spacetime $M$ is a symmetric 2-covariant tensor field $T$ such that
	\begin{equation}
	\label{eq:stress}
		T(V,V) \geq 0
	\end{equation}
	for any timelike vector $V \in T_p M \; \forall p \in M$.
\end{definition}

Note that a continuity argument shows that inequality \ref{eq:stress} also holds true if $V$ is a lightlike vector on $T_p M$.

It is commonly argued that the mathematical way to express that gravity attracts on average is
\begin{equation}
	\operatorname{Ric} (V,V) \geq 0
\end{equation}
for any timelike tangent vector $V \in T_p M$ and for all $p \in M$; that is, the Ricci tensor of a physically realistic spacetime must be an stress-energy tensor. This assumption restricts the familiy of Lorentzian metrics that can be physically significant.

\section{.}

For a given stress-energy tensor $T$, a spacetime $(M,g)$ is said to obey the Einstein equation if
\begin{equation}
	G(g) = T
\end{equation}
where $G$ is the Einstein tensor of $g$; that is,
\begin{equation}
	\operatorname{Ric} - \frac{1}{2} S g = T
\end{equation}