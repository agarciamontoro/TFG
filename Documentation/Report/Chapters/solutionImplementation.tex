\chapter{Solution implementation}

This chapter covers the implementation details, the technologies used, the different decisions made and the reasons that led us to make them.

In short, the software developed is a Python package that implements a general relativity ray tracer using the library \ac{CUDA} as the back-end, generating images of a Kerr black hole from close distances.

The primary requirement when designing and implementing the software has been the \emph{ease of use}. The Python package exposes a minimal yet powerful public \ac{API}, abstracting all \ac{CUDA}-related code and letting the user configure the properties of the black hole and the cameras placed near it.

\section{Technologies Used}

The following technologies have been used on the implementation of this software:
\begin{enumerate}
	\item Programming languages: Python and C.
	\item Parallelization library: CUDA.
	\item Documentation: Sphinx and Doxygen.
\end{enumerate}

\subsection{Python}
\subsection{CUDA}
\subsection{PyCUDA}
\subsection{Documentation: Sphinx and Doxygen}

\section{Algorithm Implementation}
\subsection{Initial Conditions Computation}
\subsection{Raytracing}

\section{CUDA Parallelization}
\subsection{Device Setup}
\subsection{CUDA Kernel}
\subsection{Fine-Tuning}