\chapter{Kerr Spacetime}
\label{chapter:kerr}

After all the mathematical background, the study of the new concepts properties and the physical introduction, we are ready to introduce the Kerr spacetime, a spacetime that models one of the most interesting and still unknown celestial bodies: the black holes.

The Kerr black hole is the centre (both metaphorically and literally) of our model: it is a celestial body, sometimes known as a black star, so massive that it warps the spacetime in such a way that even the light cannot escape of its force. These theoretical objects are the main line of investigation of both students and renowned experts.

This chapter tries to define it, comment some properties and extract the necessary information we will need to build the ray tracer. A deep study of the Kerr spacetime, however, is far beyond the scope of this work, so the origin of the metric tensor will not be described and the proofs of the results will be omitted, although the main references will be given.

Let us set up some terminology, described in \cite[Def. 1.6.2, 1.6.3]{oneill95} in order to understand the following pages.

\begin{definition}[Timelike particle]
	A future-pointing timelike curve $\alpha$ in a spacetime is known as a \emph{timelike (or material) particle}. Its mass is defined as $m = \vert \alpha' \vert$.
\end{definition}

\begin{definition}[Lightlike particle]
	A \emph{lightlike (or null) particle} $\gamma$ in a spacetime is a future-pointing lightlike geodesic. It is trivial to see that its mass is zero.
\end{definition}

\begin{definition}[Causal particle]
	A \emph{causal particle} in a spacetime is either a material particle or a lightlike particle.
\end{definition}

The tangent vector of a causal particle $\gamma$, noted as $\mathbf{p} = \gamma'$, is known as the \emph{energy-momentum four vector}, simply called \emph{momentum}.

Throughout all the chapter we choose natural units; \ie, we set the speed of light and the mass to the unity. This is the usual way of working in physics, and it will ease the notation of the following equations.

\section{Kerr Metric}

The Kerr metric is an exact solution of the vacuum Einstein field equation, that models a spacetime with an axially symmetric black hole rotating \footnote{It is important to note here that the black hole does not actually rotate: the Kerr metric is a solution to the vacuum Einstein field equation, so no mass is located on the spacetime. When we talk about the mass of the black hole, we are really talking about the mass of the colliding star that formed it. However, as the black hole curves the spacetime in such a way that any particle falling into it is forced to rotate, we can picture this as the rotation of the black hole itself.} about an axis through its centre.

The coordinate chart we are going to use to describe each event occuring on the spacetime's manifold is defined by the \ac{BL} coordinates, whose four components are noted as $(t, r, \vartheta, \varphi)$.

The basis for the tangent space of the spacetime will be noted as $(\partial_t, \partial_r, \partial_\vartheta, \partial_\varphi)$, whereas its dual basis will be noted as $(dt, dr, d\vartheta, d\varphi)$

Although, historically, the origin of these coordinates did not have any physical meaning, an observer placed at the infinity would understand them as follows:
\begin{enumerate}
	\item $t\in\R$ is the time coordinate.
	\item $(r, \vartheta, \varphi)$ are the spatial coordinates:
	\begin{enumerate}
		\item $r \in \R^+$ is the distance to the centre of the black hole.
		\item $\vartheta \in \R$ is the polar angle with respect to the centre of the black hole.
		\item $\varphi \in \R$ is the azimuthal angle with respect to the centre of the black hole, with the black hole itself rotating in the positive $\varphi$ direction.
	\end{enumerate}
\end{enumerate}

Kerr spacetime depends on two parameters that models the black hole:
\begin{enumerate}
	\item $M > 0$, which could be considered as its mass.
	\item $a \neq 0$, its angular momentum per unit of mass.
\end{enumerate}

As all units, the mass will be set to $M = 1$, and so the black hole will only be parametrized by its angular momentum, that is, how fast its rotation is.

To introduce the Kerr metric, we first need to define two functions that will be omnipresent in the study of this particular spacetime:
\begin{align}
	\rho^2 &= r^2 + a^2\cos^2\vartheta \\
	\Delta &= r^2 + -2r + a^2
\end{align}

From these equations, the Kerr metric is defined as follows.
\begin{definition}[Kerr metric]
	The \emph{Kerr metric}, described by its line element using \ac{BL} coordinates, is:
	\begin{align}
		\label{eq:kerrmetric}
		ds^2 = &-dt^2 + \rho^2\left(\frac{dr^2}{\Delta} + d\vartheta^2\right) + \left(r^2 + a^2\right)\sin^2\vartheta d\varphi^2 + \\
		\nonumber
		&+ \frac{2r}{\rho^2}\left(a\sin^2\vartheta d\varphi - dt\right).
	\end{align}
\end{definition}

From the definition, some properties of the Kerr metric can be observed \cite[Sec. 2.1]{galindo14}, \cite[pp. 58-59]{oneill95}:
\begin{enumerate}
	\item The Kerr metric is stationary, that is, it not depends on the time coordinate, $t$.
	\item The Kerr metric is axially symmetric, that is, it not depends on the azimuthal angle coordinate, $\varphi$.
	\item Since the Kerr metric does not depend on the coordinates $t$ and $\varphi$, the coordinate vector fields $\partial_t$ and $\partial_\varphi$ are Killing. This summarises the two previous properties.
	\item The Kerr metric is invariant when applying the transformation
	\begin{align*}
		t &\to -t \\
		\varphi &\to -\varphi,
	\end{align*}
	that is, travelling to the past reverses the rotation.
	\item The Kerr spacetime is asymptotically flat, that is, far from the black hole, the spacetime is flat and the coordinate $r$ can be really viewed as the distance to the black hole. The coordinates $(r, \vartheta, \varphi)$ can then, and only then, be understood as spherical coordinates.
\end{enumerate}

\section{Symmetries}

As seen before, Kerr metric is stationary and axially symmetric, from where two symmetries can be obtained using the two Killing vector fields.

If we consider $\mathbf{v}^\alpha$ the tangent vector to a geodesic $\gamma$ and we note
\[
	\xi_1 = \partial_t, \qquad \xi_2 = \partial_\varphi
\]
the two quantities $\mathbf{v}^\alpha \xi_{1\alpha}$ and $\mathbf{v}^\alpha \xi_{1\alpha}$ are conserved, and can be noted as
\begin{align}
	\label{eq:energy}
	-E &\defeq \mathbf{v}^\alpha \xi_{1\alpha} \\
	L_z &\defeq \mathbf{v}^\alpha \xi_{1\alpha}.
\end{align}

This comes from the fact 

Considering a timelike geodesic, the previous quantities can be understood as the energy of the particle and the angular momentum, both per unit mass and measured from the infinity.

Even if we consider a lightlike geodesic, this interpretation can be maintained by means of parametrizing the geodesic flow with the affine parameter of the geodesic.

The convention for these two conserved quantities will be hold until the end of the document, and will prove key when trying to obtain the equations of motion for a free fall causal particle.


\section{Horizons and Singularities}

There are whole books, as \cite{oneill95}, whose only aim is to study Kerr spacetime and, in particular, its singularities and horizons, that is, the surfaces on the spacetime which prevent the particles inside them to go out of its region.

This study is beyond the scope of this work, but a short summary of the singularities found on Kerr spacetime is developed below. All of them are explained in detail on \cite[Sec. 2.4]{galindo14}.

From \autoref{eq:kerrmetric}, one can see that the Kerr metric is singular in any of the following cases:
\begin{align}
	\rho^2 &= 0 \\
	\Delta &= 0
\end{align}

These \emph{singularity equations} throw two different types of singularities: the ones produced by the choice on the coordinate system, but which are not real singularities of the metric (i.e., that can be avoided when changing the coordinate system), and the ones that can be obtained using any coordinate system.

The first singularity, which is caused by the nature of the spacetime and not by the choice of the coordinate system, is obtained when assuming $\rho^2 = 0$, that makes the Kretschmann invariant\footnote{In Kerr spacetimes, the usual scalar curvature (see \autoref{def:scalarcurvature}) vanishes, as $Ric = 0$. However, one can define another invariant quantity in order to describe the spacetime curvature. This description is usually carried out with the help of the Kretschmann invariant, that is defined as $K = Ric_{\alpha\beta\gamma\delta}Ric^{\alpha\beta\gamma\delta}$.} divergent, and that can be physically understood as a divergent curvature of the spacetime itself.

From the second singularity equation, we obtain two coordinate singularities:
\[
	r_{\pm} \defeq M \pm \sqrt{M - a^2}
\]

These singularities are called \emph{horizons}, as it can be proved \cite[p. 15]{galindo14} that, on them, the constant-$r$ hypersurface are null. The two horizons divide the spacetime in three regions:
\begin{enumerate}
	\item The region where $r > r_+$, that can be seen as the outside of the black hole.
	\item The region where $r_- < r < r_+$. Here, any particle falling through $r_+$ is forced to reach $r_-$, reason why the hypersurface defined by $r = r_+$ is called the \emph{event horizon}.
	\item The region where $r < r_-$, which contains the spacetime singularity.
\end{enumerate}

\begin{remark}
	From this second singularity equation, one could think that $a$ should always satisfy $a^2 < M^2$. This is not exactly true, as, theoretically, black holes with $a^2 \geq M^2$, called \emph{fast and extreme Kerr black holes} can be studied. When $a^2 > M^2$, the singularity equation has no real solution and therefore no horizons exist.
	
	Although these theoretical spacetimes exist, we will always assume that $a^2 < 1 = M^2$.
\end{remark}

\section{Kerr Metric notation}

In the following pages, the Kerr metric will be repeatedly used, so its matrix, along with its inverse, is depicted here for consultation purposes.

The Kerr metric tensor matrix, described on a \ac{BL} system coordinate, is
\[
	g_{\mu\nu} = \begin{pmatrix}
		-\alpha^2 + \varpi^2\omega^2 & 0 & 0 & -\varpi^2\omega \\
		0 & \nicefrac{\rho^2}{\Delta} & 0 & 0 \\
		0 & 0 & \rho^2 & 0 \\
		-\varpi^2\omega^2 & 0 & 0 & \varpi^2
	\end{pmatrix},
\]

The inverse of the Kerr metric is then
\[
	g^{\mu\nu} = \begin{pmatrix}
		-\nicefrac{1}{\alpha^2} & 0 & 0 & -\nicefrac{\omega}{\alpha^2} \\
		0 & \nicefrac{\Delta}{\rho^2} & 0 & 0 \\
		0 & 0 & \nicefrac{1}{\rho^2} & 0 \\
		-\nicefrac{\omega}{\alpha^2} & 0 & 0 & (\nicefrac{1}{\varpi^2}) - (\nicefrac{\omega^2}{\alpha^2})
	\end{pmatrix},
\]


In both notations, the following formulas are used:
\begin{align}
	\omega &= \frac{2ar}{\Sigma^2},  \quad \varpi = \frac{\Sigma\sin\vartheta}{\rho}, \quad \alpha = \frac{\rho\sqrt{\Delta}}{\Sigma}, \quad \rho = \sqrt{r^2 + a^2\cos^2\vartheta},\nonumber\\
	\Delta &= r^2 - 2r + a^2, \textrm{ and } \Sigma = \sqrt{(r^2+a^2)^2 - a^2\Delta\sin^2\vartheta}.
	\label{eq:termdef}
\end{align}