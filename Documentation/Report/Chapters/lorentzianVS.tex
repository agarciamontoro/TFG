\chapter{Lorentzian vector spaces}
Let $V$ be a real vector space with $\mathrm{dim}\, V=n(\geq 2)$ and let $g$ be a non-degenerate
symmetric bilinear form on $V$ with index 1, i.e., $g(u,v)=0$ for all $v\in V$ implies $u=0$,
and the maximum dimension of a subspace $L$ of $V$ such that $g(u,u)\leq 0$ for all $u\in L$
with equality if and only if $u=0$ is 1. We will say that $g$ is a \emph{Lorentzian product}
and $(V,g)$ a \emph{Lorentzian vector space}. A vector $v$ is said to be \emph{spacelike}
(resp. \emph{timelike}, \emph{null}) if $g(v,v)>0$ or $v=0$ (resp. $g(v,v)<0$, $g(v,v)=0$ and
$v\neq 0$). The \emph{light cone} is the subset of $V$ consisting of all null vectors of $(V,g)$.

\vspace{2mm}

The basic tool we will use to study Lorentzian products is the following result \cite[Lemma 5.26]{oneill83}

\begin{lemma}\label{basic} If $v$ is a timelike vector of $(V,g)$ then we have the orthogonal decomposition
\[
V=\mathrm{Span}\{v\}\oplus v^{\perp},
\]
where $v^{\perp}=\{u\in V\, : \, g(u,v)=0\}(=\mathrm{Span}\{v\}^{\perp})$. Moreover, the restriction
of $g$ on $v^{\perp}$ is positive definite (i.e. $g_{{\mid}_{v^{\perp}}}$ is a Euclidean product) and
the restriction of $g$ to $\mathrm{Span}\{v\}$ is negative definite (i.e. $(-g)_{{\mid}_{\mathrm{Span}\{v\}}}$
is a Euclidean product).
\end{lemma}

\begin{remark}
    This result is not true if $v$ is assumed to be null. In fact, in this case $v\in v^{\perp}$.
    On the other hand, for $v\neq 0$ spacelike, the analogous decomposition holds but if dim$V\geq 3$ then $g_{{\mid}_{v^{\perp}}}$ is also Lorentzian.
\end{remark}

\begin{remark}\label{classic_Schwarz}
    The classical Euclidean geometry holds in positive definite subspaces of a Lorentzian vector space; in particular, the Schwarz inequality holds true in $v^{\perp}$, for any timelike vector $v$,
    \[
        \mid g(u,w)\mid \leq \sqrt{g(u,u)}\sqrt{g(w,w)}
    \]
    for all $u,w\in v^{\perp}$, with equality if and only if $u$ and $w$ are linearly dependent.
\end{remark}

One sees that any two timelike vectors of $(V,g)$ are never orthogonal.
However, for null vectors we have  \cite[Cor. 1.1.5]{sachs77}, \cite[p. 155]{oneill83}.

\begin{proposition} Given two null vectors $x,y$ of a Lorentzian vector space $(V,g)$ we have
$g(x,y)=0$ if and only if $x$ and $y$ are linearly dependent.
\end{proposition}

\begin{proof}
    Using Lemma \ref{basic} we write $V=\mathrm{Span}\{v\}\oplus v^{\perp},$ where $g(v,v)=-1$. Therefore we have
    \begin{align*}
        x=av+u, \quad u\in v^{\perp}, \quad u\neq 0, \quad g(u,u)=a^2, \quad (a\neq 0),\\
        y=bv+w, \quad w\in v^{\perp}, \quad w\neq 0, \quad g(v,v)=b^2, \quad (b\neq 0).
    \end{align*}
    If $g(x,y)=0$ then $\mid g(u,w)\mid = \sqrt{g(u,u)}\sqrt{g(w,w)}$ and therefore $u=kw$ for some $k\in \R$. In fact, $k=a/b$. The converse is trivial.
\end{proof}

Let $\mathcal{T}(V,g)$ be the subset of $V$ consisting of all timelike vectors of $(V,g)$. For each $v\in \mathcal{T}(V,g)$ we put
\[
C(v)=\{u\in\mathcal{T}(V,g)\, : \, g(u,v)<0 \}.
\]
Observe that $v\in C(v)$ for any $v\in \mathcal{T}(V,g)$, hence $C(v)\neq \emptyset$. Moreover, given another $w\in \mathcal{T}(V,g)$
we know $g(v,w)\neq 0$. Therefore, either $w\in C(v)$ or $w\in C(-v)$ and so
\[
\mathcal{T}(V,g)=C(v) \cup C(-v)
\]
for any $v\in \mathcal{T}(V,g)$. We call $C(v)$ the \emph{time cone} defined by $v$. The following result characterizes
when two timelike vectors lie to the same time cone \cite[Lemma 5.29]{oneill83}

\begin{lemma}\label{time_cone}
Given $u,v\in \mathcal{T}(V,g)$, they belong to the same time cone if and only if $g(u,v)<0$.
\end{lemma}

\begin{proof}
    Clearly, if $g(u,v)<0$ then $u\in C(v)$, and so $u$ and $v$ belong to the time cone defined by $v$. Conversely, using again Lemma \ref{basic} we write
    \begin{align*}
        u=aw+y, \quad g(w,w)=-1, \quad y\in w^{\perp}, \quad a=-g(u,w)>0,//
        v=bw+z, \quad g(w,w)=-1, \quad z\in w^{\perp}, \quad b=-g(v,w)>0.
    \end{align*}
    Taking into account
    \[
    \sqrt{g(y,y)}<a \quad \mathrm{and} \quad \sqrt{g(z,z)}<b,
    \]
    we have
    \[
    g(u,v)\leq -ab+\mid g(y,z)\mid \leq -ab + \sqrt{g(y,y)}\sqrt{g(z,z)}<-ab+ab=0.
    \]
\end{proof}

\begin{corollary}\label{convexity}
Given $w\in\mathcal{T}(V,g)$, $u,v\in C(w)$ and $a,b\in \R$, $a,b\geq 0$, $a^2+b^2 \neq 0,$ we have $au+bv\in C(w)$.
\end{corollary}

\begin{proof}
    Compute $g(au+bv,w)=a g(u,w)+ b g(v,w)<0$ and use previous result.
\end{proof}

\begin{remark}
    Time cones are then convex subsets of $V$. On the other hand, note that if $u\in C(v)$ then $C(u)=C(v)$.
\end{remark}

We have seen in Remark \ref{classic_Schwarz} that the classical Schwarz inequality holds true in positive definite subspaces of a Lorentzian vector space. In general, this inequality does not hold for any pair of vectors. However, if we pay attention only on timelike vectors we have the so-called \emph{wrong-way Schwarz inequality} \cite[Prop. 5.30]{oneill83}

\begin{proposition}\label{wrong_way}
    For any $u,v \in \mathcal{T}(V,g)$ we have
    \[
        \mid g(u,v) \mid \geq \sqrt{-g(u,u)}\sqrt{-g(v,v)}
    \]
    and equality holds if and only if $u$ and $v$ are linearly dependent.
\end{proposition}

\begin{proof}
    We write using Lemma \ref{basic}
    \[
        u=av+x, \quad a\in \R, \quad x\in v^{\perp},
    \]
    and therefore
    \[
        g(u,v)^2=a^2g(v,v)^2, \quad \quad g(u,u)=a^2g(v,v)+g(x,x)<0,
    \]
    which gives
    \[
        g(u,v)^2=\{g(u,u)-g(x,x)\}g(v,v) \geq g(u,u)g(v,v),
    \]
    and equality holds if and only if $g(x,x)=0$, i.e. $x=0$, which means $u=av$.
\end{proof}

Taking into account Lemma \ref{time_cone}, if two timelike vectors $u$ and $v$ belong to the same time cone we have, from Proposition \ref{wrong_way},
\[
    -g(u,v) \geq \sqrt{-g(u,u)}\sqrt{-g(v,v)},
\]
and therefore, there exists a unique $\theta \in \R$, $\theta \geq 0$, such that
\[
    \cosh \theta =\frac{-g(u,v)}{\sqrt{-g(u,u)}\sqrt{-g(v,v)}},
\]
which is called the \emph{hyperbolic angle} between $u$ and $v$.

\vspace{2mm}

As a consequence of Proposition \ref{wrong_way}, taking in mind Lemma \ref{time_cone} we get the so-called \emph{wrong-way Minkowski inequality} \cite[Cor. 5.31]{oneill83}

\begin{corollary}\label{Minkowski}
    For any $u,v \in \mathcal{T}(V,g)$ in the same time cone we have
    \[
        \sqrt{-g(u+v,u+v)} \geq \sqrt{-g(u,u)}+\sqrt{-g(v,v)}
    \]
    and equality holds if and only if $u$ and $v$ are linearly dependent.
\end{corollary}


Now, we will explain a genuine notion in Lorentzian geometry. Let $(V,g)$ be a Lorentzian
vector space and let $B=(v_1,..,v_{n-1},v_n)$ be a basis of $V$ such that
\[
M_B(g) = \left( \begin{array}{cc}
I_{n-1} & 0  \\
0 & -1
\end{array} \right),
\]
i.e. any two vectors of $B$ are $g$-orthogonal, each $v_j$, $1\leq j \leq n-1$, is unitary spacelike and
$v_n$ is unitary timelike. The basis $B$ is called an \emph{orthonormal basis} of $(V,g)$. We will say
that two orthonormal basis $B$ and $B'$ of $(V,g)$ define the same \emph{time orientation} of $(V,g)$
when $g(v_n,v'_n)<0$, i.e. if and only if $v_n$ and $v'_n$ lie in the same time cone of $(V,g)$.

\vspace{2mm}

This clearly defines an equivalence relation on the set of all orthonormal basis of $(V,g)$, which possesses
only two equivalence classes (independently of the dimension of $V$): the class defined by $B=(v_1,..,v_{n-1},v_n)$
and the class defined by $\tilde{B}=(v_1,..,v_{n-1},-v_n)$. A \emph{time orientation} on $(V,g)$ is each of these
two classes. Equivalently, a time orientation on $(V,g)$ is given by each time cone of $(V,g)$.

\vspace{2mm}

It should be noticed that the notion of an orientation on $V$ does not depend on $g$; i.e. it is not
a metric concept, whereas the notion of time orientation on $(V,g)$ is defined making use of the Lorentzian
product $g$, although a time orientation does not change if we change $g$ to a conformally related Lorentzian product
$ag$, $a\in \R$, $a>0$.

\vspace{2mm}

We end this section explaining some topological remarks. If $V$ is an $n$-dimensional vector space and $B$ is a basis of $V$, then we have a linear isomorphism $$b_B : V \longrightarrow \R^n,$$ defined by $b_B(v)=(a_1,..,a_n)$, where $(a_1,..,a_n)$ are the coordinates of $v$ in $B$. By using $b_B$ a topology $\mathfrak{T}_B$ can be defined in $V$ in such a way that the $\mathfrak{T}_B$-open subsets of $V$ are $b_B^{-1}(O)$, where $O$ is an open subset of $\R^n$. It is not difficult to see that $\mathfrak{T}_B = \mathfrak{T}_{B'}$ for any basis $B,B'$ of $V$. Endowed with this topology, the function $V \longrightarrow \R$, given by $v \mapsto g(v,v)$ is continuous. Hence $\mathcal{T}(V,g)$ is an open subset of $V$ and each time cone is also an open subset of $V$. Moreover, the set of nonzero spacelike vectors $\{v\in V \, : \, g(v,v)>0 \}$ is an open subset of $V$ and the light cone with the zero vector $\{v\in V \, : \, g(v,v)=0 \}$ is a closed subset of $V$. Finally, every null vector can be obtained as the limit of a sequence of timelike vectors as well of spacelike vectors (or both types of vectors). Thus, $\{v\in V \, : \, g(v,v)<0 \} \cup \{v\in V \, : \, g(v,v)>0 \}$
is a dense subset of $V$.
