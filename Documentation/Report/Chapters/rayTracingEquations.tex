\chapter{Ray-tracing equations}

The Kerr metric is
\[
	g_{\mu\nu} = \begin{pmatrix}
		-\alpha^2 + \varpi^2\omega^2 & 0 & 0 & -\varpi^2\omega \\
		0 & \nicefrac{\rho^2}{\Delta} & 0 & 0 \\
		0 & 0 & \rho^2 & 0 \\
		-\varpi^2\omega^2 & 0 & 0 & \varpi^2
	\end{pmatrix},
\]
with inverse
\[
	g^{\mu\nu} = \begin{pmatrix}
		-\nicefrac{1}{\alpha^2} & 0 & 0 & -\nicefrac{\omega}{\alpha^2} \\
		0 & \nicefrac{\Delta}{\rho^2} & 0 & 0 \\
		0 & 0 & \nicefrac{1}{\rho^2} & 0 \\
		-\nicefrac{\omega}{\alpha^2} & 0 & 0 & (\nicefrac{1}{\varpi^2}) - (\nicefrac{\omega^2}{\alpha^2})
	\end{pmatrix},
\]
where
\begin{align}
	\omega &= \frac{2ar}{\Sigma^2},  \quad \varpi = \frac{\Sigma\sin\vartheta}{\rho}, \quad \alpha = \frac{\rho\sqrt{\Delta}}{\Sigma}, \quad \rho = \sqrt{r^2 + a^2\cos^2\vartheta},\nonumber\\
	\Delta &= r^2 - 2r + a^2, \textrm{ and } \Sigma = \sqrt{(r^2+a^2)^2 - a^2\Delta\sin^2\vartheta}.
	\label{eq:termdef}
\end{align}

In order to obtain the geodesic equations we need to note that we have four first integrals of motion \fixme{Define integral of motion}:
\begin{itemize}
	\item The killing vector \fixme{Define killing vector} $\partial_t$, which raises the conserved quantity
	\[
		g_{\mu\nu}(\partial_t)^\mu u^\nu = (\partial_t)^\mu p_\mu = p_t = -E
	\]
	\item The killing vector $\partial_\phi$, which raises the conserved quantity
	\[
		g_{\mu\nu}(\partial_\phi)^\mu u^\nu = (\partial_\phi)^\mu p_\mu = p_\phi = L_z
	\]
	\item The Carter's constant, which can be written as
	\begin{equation}
		\label{eq:carter}
		Q = p_\vartheta^2 + \cos^2\vartheta \left( a^2 \left( \mu^2 - E^2 \right) + \frac{L_z^2}{\sin^2\vartheta} \right)
	\end{equation}
	\item The modulus of the tangent vector $p_\mu$, which is
	\begin{equation}
		\label{eq:modulus}
		p_\mu p_\nu g^{\mu\nu} = -\mu^2,
	\end{equation}
	that is,
	\[
		\mu^2 + \frac{p_\vartheta^2 + p_r^2 \Delta}{\rho^2} + \frac{L_z^2}{\varpi^2} = \frac{E - L_z \omega}{\alpha^2}
	\]
\end{itemize}

In order to obtain the classic equations we need to know the relation between $u^\mu$ and $p_\mu$, which is simply
\begin{equation}
	\label{eq:relation}
	u^\mu = g^{\mu\nu}p_\nu,
\end{equation}
where their components will be denoted as
\[
	u^\mu = (\dot{t}, \dot{r}, \dot{\vartheta}, \dot{\varphi}), \qquad u_\mu = p_\mu = (p_t, p_r, p_\vartheta, p_\varphi).
\]

Therefore, \autoref{eq:relation} written on the components noted above, gives us four equations:
\begin{align}
	\label{eq:initt}
	\dot{t} &= \frac{E}{\alpha^2} - \frac{L_z \omega}{\alpha^2} \\
	\label{eq:initr}
	\dot{r} &= \frac{p_r \Delta}{\rho^2} \\
	\label{eq:inittheta}
	\dot{\vartheta} &= \frac{p_\vartheta}{\rho^2} \\
	\label{eq:initphi}
	\dot{\varphi} &= \frac{E \omega}{\alpha^2} + L_z\left( \frac{1}{\varpi^2} - \frac{\omega^2}{\alpha^2} \right)
\end{align}

Our goal now is to write the previous equations in terms of the constants $E$, $L_z$, $Q$ and $\mu$. Let us see each equation separately.

\section{$\dot{r}$ Equation}

Using \autoref{eq:modulus}, we can find the value of $p_r$:
\[
	p_r^2 = \left( \frac{(E - L_z\omega)^2}{\alpha^2} - \frac{L_z^2}{\varpi^2} - \frac{p_\vartheta^2}{\rho^2} - \mu^2 \right) \frac{\rho^2}{\Delta}.
\]

We can now substitute $p_r$ for its value in the squared version of \autoref{eq:initr}:
\begin{equation}
	\label{eq:r1}
	\dot{r}^2 = \Delta \left( -\eqnote{\dot{\vartheta}}{from $p_\vartheta = \rho^2\dot{\vartheta}$}^2 + \frac{E^2}{\alpha^2\rho^2} - \frac{\mu^2}{\rho^2} - \frac{2L_zE\omega}{\alpha^2\rho^2} + \frac{L_z^2\omega^2}{\alpha^2\rho^2} - \frac{L_z^2}{\rho^2\varpi^2} \right)
\end{equation}

If we write \autoref{eq:r1} in the form
\[
	\rho^2 \dot{r} = \sqrt{R},
\]
the function $R$ has the expression
\begin{align}
	\nonumber
	R &= \rho^4 \left( -\frac{\Theta}{\rho^4} + \frac{(E^2 - L_z\omega)^2}{\alpha^2\rho^2} - \frac{\mu^2}{\rho^2} - \frac{L_z^2}{\rho^2\varpi^2} \right) = \\
	&= \Delta \left( -\Theta + \frac{(E-L_z\omega)^2\rho^2}{\alpha^2} - \mu^2\rho^2 - \frac{L_z^2 \rho^2}{\varpi^2} \right).
	\label{eq:R1}
\end{align}

Notice that
\[
	\frac{\rho^2}{\alpha^2} = \frac{\Sigma^2}{\Delta}, \qquad \frac{\rho^2}{\varpi^2} = \frac{\rho^4}{\Sigma^2 \sin^2\vartheta}.
\]

Let us now substitute $\Theta$ by its definition (\autoref{eq:deftheta}) on \autoref{eq:R1}, from which we obtain:

\begin{align}
	R =\,& \Delta \biggl( -Q + \cos^2\vartheta\left( \frac{L_z^2}{\sin^2\vartheta} + a^2 (\mu^2 - E^2) \right) + \frac{\Sigma^2}{\Delta}(E - L_z\omega)^2 - \nonumber \\
	&- \mu^2\rho^2 - \frac{L_z^2\rho^4}{\Sigma^2\sin^2\vartheta} \biggr) = \nonumber \\
	=\,& \Delta \biggl( -Q + \cos^2\vartheta\left( \frac{L_z^2}{\sin^2\vartheta} + a^2 (\mu^2 - E^2) \right) + \nonumber \\
	&+ \frac{\Sigma^2}{\Delta}\left( E^2 + L_z^2\left( \frac{2ar}{\Sigma^2} \right)^2 - 2EL_z\frac{2ar}{\Sigma^2} \right)  - \mu^2\rho^2 - \nonumber \\
	&- \frac{L_z^2}{\sin^2\vartheta}\frac{\rho^4}{(r^2 + \omega)^2 - a^2\Delta\sin^2\vartheta} \biggr)
\end{align}

This can be simplified in order to get a more readable expression, although the work will be somewhat cumbersome.

Let us start by simplifying the previous expression by actually making the product by the factorised $\Delta$ and the inner $\Sigma^2$:

\begin{align}
	R =\,& -Q\Delta + \Delta\cos^2\vartheta\left( \frac{L_z^2}{\sin^2\vartheta} + a^2 (\mu^2 - E^2) \right) + \Sigma^2 E^2 + \nonumber \\
	&+ \frac{L_z^2 (2ar)^2}{\Sigma^2} - 2 E L_z 2ar  - \mu^2\rho^2\Delta - \frac{L_z^2}{\sin^2\vartheta}\frac{\Delta\rho^4}{\Sigma^2}
	\label{eq:R2}
\end{align}

Rearranging \autoref{eq:R2}, we obtain
\begin{align}
	R =& \overbrace{\frac{L_z^2 \left(4a^2r^2 - \frac{\Delta\rho^4}{\sin^2\vartheta}\right)}{\Sigma^2}}^{\text{(\dag)}} - 4 a r E L_z - \mu^2 \rho^2 \Delta + \Sigma^2 E^2 - Q \Delta + \nonumber \\ 
	&+ \Delta\cos^2\vartheta\left(\frac{L_z^2}{\sin^2\vartheta} + a^2\left(\mu^2 - E^2\right)\right).
\end{align}

Let us focus now on (\dag):

\begin{align}
	(\dag) =\,& \frac{L_z^2 \left(4a^2r^2 - \frac{\Delta\rho^4}{\sin^2\vartheta}\right)}{\Sigma^2} = \frac{L_z^2 \left(4a^2r^2\sin^2\vartheta - \rho^4\Delta \right)}{\sin^2\vartheta\left(\left(r^2+a^2\right)^2 - a^2\Delta\sin^2\vartheta\right)} \nonumber \\
	=\,& \frac{\Biggl\{L_z^2\left(4a^2r^2\sin^2\vartheta+\left(r^2+2r-a^2\right)\left(r^2+\omega^2\cos^2\vartheta\right)^2\right)\Biggr\}(\ddag)}{\sin^2\vartheta\left(\left(r^2+a^2\right)^2-a^2\Delta\sin^2\vartheta\right)}
\end{align}

Let us try to simplify (\ddag), the numerator of (\dag), first:

\begin{align}
	(\ddag) =\,& L_z^2 \Biggl( -r^6 + 2r^6 + r^4 \left( -a^2 - 2a^2\cos^2\vartheta \right) + r^3\left(4 a^2 \cos^2\vartheta \right) + \nonumber \\
	&+ r^2 \left( -2a^4\cos^2\vartheta - a^4\cos^4\vartheta + 4a^2\sin^2\vartheta\right) + r\left(2a^4\cos^4\vartheta\right) - \nonumber \\
	&- a^6\cos^4\vartheta \Biggr) = L_z^2 \left( \left(r^2 + a^2\right)^2 - a^2 \left(r^2 - 2r + a^2\right)\sin^2\vartheta \right) \cdot \nonumber \\
	&\cdot \left( -\frac{a^2}{2} + 2r - r^2 - \frac{1}{2}a^2\cos^2\vartheta + \frac{1}{2a^2\sin^2\vartheta} \right) = \nonumber\\
	=\,& L_z^2 \Sigma^2 \left( -\frac{a^2}{2} + 2r - r^2 - \frac{1}{2}a^2\cos^2\vartheta + \frac{1}{2a^2\sin^2\vartheta} \right)
\end{align}

Then, the term (\dag) becomes:
\[
	(\dag) = \frac{L_z^2 \left( -\frac{a^2}{2} + 2r - r^2 - \frac{1}{2}a^2\cos^2\vartheta + \frac{1}{2}a^2\sin^2\vartheta\right) }{\sin^2\vartheta},
\]
and so the function $R$ can be now written as:

\begin{align}
	R =& \frac{L_z^2 \left( -\frac{a^2}{2} + 2r - r^2 - \frac{1}{2}a^2\cos^2\vartheta + \frac{1}{2}a^2\sin^2\vartheta\right) }{\sin^2\vartheta} - \nonumber\\
	&- 4 a r E L_z - \mu^2 \rho^2 \Delta + \Sigma^2 E^2 - Q \Delta + \nonumber \\ 
	&+ \Delta\cos^2\vartheta\left(\frac{L_z^2}{\sin^2\vartheta} + a^2\left(\mu^2 - E^2\right)\right).
\end{align}

Substituting $\Sigma$ and $\Delta$ by their values, defined on equations \ref{eq:termdef}, we obtain
\begin{align}
	R =\,& \frac{a^2L_z^2}{2} - \Delta Q - 4aL_zrE + a^4E^2 + 2a^2r^2E^2 + r^4E^2 - a^2r^2\mu^2 + \nonumber \\
	&+ 2r^3\mu^2 - r^4\mu^2 + \nonumber\\
	&+ cos^2\vartheta\left(-a^4E^2 + 2a^2rE^2 - a^2r^2E^2 \right) + \nonumber \\
	&+ \sin^2\vartheta\left( -a^4E^2 + 2a^2rE^2 - a^2r^2E^2 \right) + \nonumber\\
	&+ \cot^2\vartheta\left( \frac{a^2L_z^2}{2} - 2L_z^2r + L_z^2r^2 \right) + \nonumber \\
	&+ \csc^2\vartheta\left( -\frac{a^2L_z^2}{2} + 2L_z^2r - L_z^2r \right).
\end{align}

We can simplify the last two pairs of terms using that $\sin^2 \vartheta + \cos^2 \vartheta = 1$ and that $\csc^2\vartheta - \cot^2\vartheta = 1$:
\begin{align}
	R =\,& \frac{a^2L_z^2}{2} - \Delta Q - 4aL_zrE + a^4E^2 + 2a^2r^2E^2 + r^4E^2 - a^2r^2\mu^2 + \nonumber \\
	&+ 2r^3\mu^2 - r^4\mu^2 + \left(-a^4E^2 + 2a^2rE^2 - a^2r^2E^2 \right) + \nonumber \\
	&+ \left( -\frac{a^2L_z^2}{2} + 2L_z^2r - L_z^2r \right).
\end{align}

Factoring out common terms in the last two addends, using the definition of $\Delta$ and defining the function
\begin{equation}
	\label{eq:defP}
	P \defeq E(r^2 + a^2) - aL_z,
\end{equation}
we simplify a little bit more:
\begin{align}
	R =\,& \frac{a^2L_z^2}{2} - \Delta Q - 4aL_zrE + a^4E^2 + 2a^2r^2E^2 + r^4E^2 - a^2r^2\mu^2 + \nonumber \\
	&+ 2r^3\mu^2 - r^4\mu^2 + a^2E^2\left(-a^2 + 2r - r^2 \right) + L_z^2\left( -\frac{a^2}{2} + 2r - r^2 \right) = \nonumber \\
	=\,& \frac{a^2L_z^2}{2} - \Delta Q - 4aL_zrE + a^4E^2 + 2a^2r^2E^2 + r^4E^2 - a^2r^2\mu^2 + \nonumber \\
	&+ 2r^3\mu^2 - r^4\mu^2 - a^2E^2\Delta + L_z^2\Delta + \frac{a^2L_z^2}{2} = \nonumber \\
	=\,& \overbrace{a^2L_z^2 + a^4E^2 + 2a^2E^2r^2 + r^4E^2 - 2L_zr^2Ea - 2a^3L_zE}^{P^2} + \nonumber\\
	&+ 2L_zr^2Ea + 2a^3L_zE - \Delta Q - 4aL_zrE - a^2E^2\Delta - L_z^2\Delta - \nonumber \\
	&- a^2r^2\mu^2 + 2r^3\mu^2 - r^4\mu^2 - \Delta Q = \nonumber \\
	=\,& P^2 + \mu^2\overbrace{\left( -r^4 + 2r^3 - a^2r^2 \right)}^{\Delta r^2} - L_z^2\Delta - (a^2E^2)\Delta + 2L_zEar^2 + \nonumber\\
	&+ 2a^2L_zE - 4aL_zrE - Q\Delta = \nonumber\\
	=\,& P^2 + r^2\mu^2\Delta + L_z^2\Delta - a^2E^\Delta - 2L_zaE\left(2r - r^2 - a^2\right) - Q\Delta = \nonumber\\
	=\,& P^2 + r^2\mu^2\Delta - L_z^2 \Delta - a^2E^2\Delta - 2L_zaE\Delta - Q\Delta = \nonumber \\
	=\,& P^2 - \Delta \left( r^2\mu^2 + Q + \left(L_z - aE \right)^2 \right)
\end{align}

\section{$\dot{\vartheta}$ Equation}

Using the definition of the Carter constant (\autoref{eq:carter}), we can directly find the value of $p_\vartheta$
\begin{align}
	p_\vartheta^2 &= Q + a^2E^2\cos^2\vartheta - a^2\mu^2\cos^2\vartheta - \frac{L_z^2}{\sin^2\vartheta}\cos^2\vartheta = \\
	&= Q - \cos^2\vartheta\left( a^2\left(\mu^2 - E^2 \right) `\frac{L_z^2}{\sin^2\vartheta} \right).
\end{align}

Therefore,\autoref{eq:inittheta} is equivalent to the following one:
\begin{equation}
	\rho^2 \dot{\vartheta} = \sqrt{\Theta},
\end{equation}
where $\Theta$ is defined as
\begin{equation}
	\label{eq:deftheta}
	\Theta = Q - \cos^2\vartheta \left( \frac{L_z^2}{\sin^2\vartheta} + \omega^2\left(\mu^2 - E^2 \right)\right).
\end{equation}

\section{$\dot{\varphi}$ Equation}
\[
	\rho^2 \dot{\varphi} = \frac{E2ar}{\Delta} + L_z\left( \frac{\rho^4}{\Sigma^2\sin^2\vartheta} - \frac{4a^2r^2}{\Sigma^2\Delta} \right)
\]
