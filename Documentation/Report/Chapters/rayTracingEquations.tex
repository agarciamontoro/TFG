\chapter{Equations of Motion}

This chapter aims to find the most computationally stable equations of motion for $\gamma$, a free falling lightlike particle moving on a Kerr spacetime.

The classical equations derived from the definition of geodesic in terms of the Christoffel symbols,
\[
	\frac{d^2x^k}{dt^2} + \Gamma^k_{ij} \frac{d x^i}{dt} \frac{d x^j}{dt} = 0 \quad k = 1, \dots, n,
\]
are the first stop in our journey to find an equation system that suits our needs.

These equations, although really interesting when, for example, one cannot obtain analytic expression for the differential equations, have some flaws that will force us to discard them:
\begin{itemize}
	\item It is a second order system of differential equations, which does not really fit the numerical algorithm to integrate them: when converting the system to first order, we double the number of equations to solve.
	\item The conserved quantities of the system, which will prove key in the computation of the final equations, can be obtained from these equations as well as for the versions we will see later. However, this system has a problem: it is very difficult to get the conserved quantities \emph{into} the equations.
\end{itemize}

Having discarded this obvious system of equations, we will look for the classical Hamiltonian formulation \cite[Sec. 33.5]{thorne73}, that gives us a first order system with the conserved quantities in the equations. These quantities are derived from the Hamiltonian formulation, but not the Hamiltonian nor its derived equations will be used here, as the final system equations will be obtained with a quadrature methods. In this formulation, however, another problem arises, as a pair of square roots appear on the right hand side of the equations. This problem will be analysed in the corresponding section.

Finally, we will be able to get rid of those square roots, simply by defining the Hamiltonian and using a version of it that will ease our analytical computations. This approach will lead us to a first order system with conserved quantities and which will be numerically-friendly. This is the approach followed on \cite{thorne15}.

From now on, and valid throughout this chapter, let $\gamma$ be a lightlike particle, whose tangent vector components, expressed in Boyer-Lindquist coordinates, are
\begin{equation}
	\label{eq:blcoord1}
	\mathbf{v}^\alpha = (\dot{t}, \dot{r}, \dot{\vartheta}, \dot{\varphi}),
\end{equation}
whereas its covariant equivalent version, the momentum, is noted as
\begin{equation}
	\label{eq:blcoord2}
	\mathbf{p}_\alpha = \mathbf{v}_\alpha = (p_t, p_r, p_\vartheta, p_\varphi).
\end{equation}

The relation between the two quantities is obtained through the operation of lower and raising indices using the metric. Therefore, we have the two equations
\begin{align}
	\label{eq:raisep}
	\mathbf{v}^\alpha &= g^{\beta\alpha} \mathbf{p}_\beta, \\
	\label{eq:lowerv}
	\mathbf{p}_\alpha &= g_{\beta\alpha} \mathbf{v}^\beta.
\end{align}

\section{Classical Hamiltonian Formulation}

It is known \cite[Sec. 33.5]{thorne73} that the geodesic equation $\gamma'' = 0$ is equivalent to the system
\begin{equation*}
	\frac{dx^\mu}{d\lambda} = \pd{\mathcal{H}}{p_\mu}, \qquad
	\frac{dp_\mu}{d\lambda} = - \pd{\mathcal{H}}{x^\mu}.
\end{equation*},
where $\lambda$ is an affine parameter such that $d/d\lambda = \mathbb{p}$ and where
\[
	\mathcal{H} = \frac{1}{2} g^{\mu\nu} p_\mu p_\nu
\]
is the usual Hamiltonian: a half of the square of the velocity.

This Hamiltonian formalism let us obtain four first integrals of motion\fixme{Define integral of motion}\cite[pp. 898-899]{thorne73}, that come as a result from the stationary and axial symmetry of the geometry of Kerr spacetimes, from the nearly miraculous work of Carter when computing the constant named after him and from the modulus of the momentum:
\begin{itemize}
	\item The killing vector $\partial_t$ raises the conserved quantity
	\[
		g_{\alpha\beta}(\partial_t)^\alpha u^\beta = (\partial_t)^\alpha p_\alpha = p_t = -E,
	\]
	\item Similarly, we can obtain another conserved quantity from the killing vector $\partial_\phi$:
	\[
		g_{\alpha\beta}(\partial_\phi)^\alpha u^\beta = (\partial_\phi)^\alpha p_\alpha = p_\phi = L_z,
	\]
	\item The Carter's constant, which can be written as
	\begin{equation}
		\label{eq:carter}
		Q = p_\vartheta^2 + \cos^2\vartheta \left( a^2 \left( \mu^2 - E^2 \right) + \frac{L_z^2}{\sin^2\vartheta} \right).
	\end{equation}
	\item The modulus of the tangent vector $p_\alpha$, which is
	\begin{equation}
		\label{eq:modulus}
		p_\alpha p_\beta g^{\alpha\beta} = -\mu^2,
	\end{equation}
	that is,
	\[
		\mu^2 + \frac{p_\vartheta^2 + p_r^2 \Delta}{\rho^2} + \frac{L_z^2}{\varpi^2} = \frac{E - L_z \omega}{\alpha^2}
	\]
\end{itemize}

%\begin{equation}
%	\begin{cases}
%		p_t &= -\dot{t}\alpha^2 - \dot{\varphi}\omega\varpi^2 + \dot{t}\omega^2\varpi^2 \\
%		p_r &= \frac{\dot{r}\rho^2}{\Delta} \\
%		p_\vartheta &= \dot{\vartheta}\rho^2 \\
%		p_\varphi &= \dot{\varphi}\varpi^2 - \dot{t}\omega\varpi^2
%	\end{cases}
%\end{equation}


Using the constants of motion, the relation in \autoref{eq:raisep} and the components notation used in equations \ref{eq:blcoord1} and  \ref{eq:blcoord2}, we obtain four equations:
\begin{align}
	\label{eq:initt}
	\dot{t} &= \frac{E}{\alpha^2} - \frac{L_z \omega}{\alpha^2} \\
	\label{eq:initr}
	\dot{r} &= \frac{p_r \Delta}{\rho^2} \\
	\label{eq:inittheta}
	\dot{\vartheta} &= \frac{p_\vartheta}{\rho^2} \\
	\label{eq:initphi}
	\dot{\varphi} &= \frac{E \omega}{\alpha^2} + L_z\left( \frac{1}{\varpi^2} - \frac{\omega^2}{\alpha^2} \right)
\end{align}

From this direct computation, we can express all four equations in terms of the constants $E$, $L_z$, $Q$ and $\mu$, which yields the following result \cite[p. 899]{thorne73}.

\begin{theorem}
	\label{theo:firsteqs}
	A free falling lightlike particle $\gamma$ satisfies:
	\begin{align}
		\label{eq:teq}
		\rho^2 \dot{t} &=-( aE\sin^2\vartheta - L_z) + \frac{r^2+a^2}{\Delta}P \\
		\label{eq:req}
		\rho^2 \dot{r} &= \sqrt{R} \\
		\label{eq:thetaeq}
		\rho^2 \dot{\vartheta} &= \sqrt{\Theta} \\
		\label{eq:phieq}
		\rho^2 \dot{\varphi} &=-( aE - \frac{L_z}{\sin^2\vartheta}) + \frac{a}{\Delta}P,
	\end{align}
	where $R$ and $\Theta$ are the functions defined as follows:
	\begin{align}
		\label{eq:defR}
		R &\defeq P^2 - \Delta \left( r^2\mu^2 + Q + \left(L_z - aE \right)^2 \right), \\
		\label{eq:defTheta}
		\Theta &\defeq Q - \cos^2\vartheta \left( \frac{L_z^2}{\sin^2\vartheta} + \omega^2\left(\mu^2 - E^2 \right)\right),
	\end{align}
	with $P$ an auxiliary function defined as
	\begin{equation}
		\label{eq:defP}
		P \defeq E(r^2 + a^2) - aL_z.
	\end{equation}
\end{theorem}

Before proving the theorem, it is interesting to analyse these equations. At first sight, the system look like a good candidate for our purposes, but there is a subtle problem that has to be managed: the square roots on equations \ref{eq:req} and \ref{eq:thetaeq}.

This will make the numeric computation more difficult, as it will force us to continuously check the sign of the derivative on the \emph{turning points}. At these points, we will have to decide which branch to continue the computation on.

This can be handled, but not without effort and a particular care on the numerical side of our algorithm: the turning points would be a great source of errors that we want to avoid.

Is for this reasoning that we continue looking for a better system of equations on \autoref{sec:variational}.

Let us however prove this theorem, as the work done here will not be in vain: both equations \ref{eq:req} and \ref{eq:thetaeq} will be used in the following section to obtain the final system of equations.

\begin{proof}
	We will only prove equations \ref{eq:req} and \ref{eq:thetaeq}, as the computation of the other two is really similar and does not contribute anything interesting. Furthermore, only the $\dot{r}$ and $\dot{\vartheta}$ equations will prove to be useful on the next section of this chapter.
	
	\subsection*{$\dot{r}$ Equation}
	
	Using \autoref{eq:modulus}, we can find the value of $p_r$:
	\[
	p_r^2 = \left( \frac{(E - L_z\omega)^2}{\alpha^2} - \frac{L_z^2}{\varpi^2} - \frac{p_\vartheta^2}{\rho^2} - \mu^2 \right) \frac{\rho^2}{\Delta}.
	\]
	
	We can now substitute $p_r$ for its value in the squared version of \autoref{eq:initr}:
	\begin{equation}
	\label{eq:r1}
	\dot{r}^2 = \Delta \left( -\eqnote{\dot{\vartheta}}{from $p_\vartheta = \rho^2\dot{\vartheta}$}^2 + \frac{E^2}{\alpha^2\rho^2} - \frac{\mu^2}{\rho^2} - \frac{2L_zE\omega}{\alpha^2\rho^2} + \frac{L_z^2\omega^2}{\alpha^2\rho^2} - \frac{L_z^2}{\rho^2\varpi^2} \right)
	\end{equation}
	
	If we write \autoref{eq:r1} in the form
	\[
	\rho^2 \dot{r} = \sqrt{\widehat{R}},
	\]
	the function $\widehat{R}$ has the expression
	\begin{align}
	\nonumber
	\widehat{R} &= \rho^4 \left( -\frac{\Theta}{\rho^4} + \frac{(E^2 - L_z\omega)^2}{\alpha^2\rho^2} - \frac{\mu^2}{\rho^2} - \frac{L_z^2}{\rho^2\varpi^2} \right) = \\
	&= \Delta \left( -\Theta + \frac{(E-L_z\omega)^2\rho^2}{\alpha^2} - \mu^2\rho^2 - \frac{L_z^2 \rho^2}{\varpi^2} \right).
	\label{eq:R1}
	\end{align}
	
	Notice that
	\[
	\frac{\rho^2}{\alpha^2} = \frac{\Sigma^2}{\Delta}, \qquad \frac{\rho^2}{\varpi^2} = \frac{\rho^4}{\Sigma^2 \sin^2\vartheta}.
	\]
	
	Let us now substitute $\Theta$ by its definition, \autoref{eq:defTheta}, on \autoref{eq:R1}, from which we obtain:
	
	\begin{align}
	\widehat{R} =\,& \Delta \biggl( -Q + \cos^2\vartheta\left( \frac{L_z^2}{\sin^2\vartheta} + a^2 (\mu^2 - E^2) \right) + \frac{\Sigma^2}{\Delta}(E - L_z\omega)^2 - \nonumber \\
	&- \mu^2\rho^2 - \frac{L_z^2\rho^4}{\Sigma^2\sin^2\vartheta} \biggr) = \nonumber \\
	=\,& \Delta \biggl( -Q + \cos^2\vartheta\left( \frac{L_z^2}{\sin^2\vartheta} + a^2 (\mu^2 - E^2) \right) + \nonumber \\
	&+ \frac{\Sigma^2}{\Delta}\left( E^2 + L_z^2\left( \frac{2ar}{\Sigma^2} \right)^2 - 2EL_z\frac{2ar}{\Sigma^2} \right)  - \mu^2\rho^2 - \nonumber \\
	&- \frac{L_z^2}{\sin^2\vartheta}\frac{\rho^4}{(r^2 + \omega)^2 - a^2\Delta\sin^2\vartheta} \biggr)
	\end{align}
	
	This can be simplified in order to get a more readable expression, although the work will be somewhat cumbersome.
	
	Let us start by simplifying the previous expression by actually making the product by the factorised $\Delta$ and the inner $\Sigma^2$:
	
	\begin{align}
	\widehat{R} =\,& -Q\Delta + \Delta\cos^2\vartheta\left( \frac{L_z^2}{\sin^2\vartheta} + a^2 (\mu^2 - E^2) \right) + \Sigma^2 E^2 + \nonumber \\
	&+ \frac{L_z^2 (2ar)^2}{\Sigma^2} - 2 E L_z 2ar  - \mu^2\rho^2\Delta - \frac{L_z^2}{\sin^2\vartheta}\frac{\Delta\rho^4}{\Sigma^2}
	\label{eq:R2}
	\end{align}
	
	Rearranging \autoref{eq:R2}, we obtain
	\begin{align}
	\widehat{R} =& \overbrace{\frac{L_z^2 \left(4a^2r^2 - \frac{\Delta\rho^4}{\sin^2\vartheta}\right)}{\Sigma^2}}^{\text{(\dag)}} - 4 a r E L_z - \mu^2 \rho^2 \Delta + \Sigma^2 E^2 - Q \Delta + \nonumber \\ 
	&+ \Delta\cos^2\vartheta\left(\frac{L_z^2}{\sin^2\vartheta} + a^2\left(\mu^2 - E^2\right)\right).
	\end{align}
	
	Let us focus now on (\dag):
	
	\begin{align}
	(\dag) =\,& \frac{L_z^2 \left(4a^2r^2 - \frac{\Delta\rho^4}{\sin^2\vartheta}\right)}{\Sigma^2} = \frac{L_z^2 \left(4a^2r^2\sin^2\vartheta - \rho^4\Delta \right)}{\sin^2\vartheta\left(\left(r^2+a^2\right)^2 - a^2\Delta\sin^2\vartheta\right)} \nonumber \\
	=\,& \frac{\Biggl\{L_z^2\left(4a^2r^2\sin^2\vartheta+\left(r^2+2r-a^2\right)\left(r^2+\omega^2\cos^2\vartheta\right)^2\right)\Biggr\}(\ddag)}{\sin^2\vartheta\left(\left(r^2+a^2\right)^2-a^2\Delta\sin^2\vartheta\right)}
	\end{align}
	
	Let us try to simplify (\ddag), the numerator of (\dag), first:
	
	\begin{align}
	(\ddag) =\,& L_z^2 \Biggl( -r^6 + 2r^6 + r^4 \left( -a^2 - 2a^2\cos^2\vartheta \right) + r^3\left(4 a^2 \cos^2\vartheta \right) + \nonumber \\
	&+ r^2 \left( -2a^4\cos^2\vartheta - a^4\cos^4\vartheta + 4a^2\sin^2\vartheta\right) + r\left(2a^4\cos^4\vartheta\right) - \nonumber \\
	&- a^6\cos^4\vartheta \Biggr) = L_z^2 \left( \left(r^2 + a^2\right)^2 - a^2 \left(r^2 - 2r + a^2\right)\sin^2\vartheta \right) \cdot \nonumber \\
	&\cdot \left( -\frac{a^2}{2} + 2r - r^2 - \frac{1}{2}a^2\cos^2\vartheta + \frac{1}{2a^2\sin^2\vartheta} \right) = \nonumber\\
	=\,& L_z^2 \Sigma^2 \left( -\frac{a^2}{2} + 2r - r^2 - \frac{1}{2}a^2\cos^2\vartheta + \frac{1}{2a^2\sin^2\vartheta} \right)
	\end{align}
	
	Then, the term (\dag) becomes:
	\[
	(\dag) = \frac{L_z^2 \left( -\frac{a^2}{2} + 2r - r^2 - \frac{1}{2}a^2\cos^2\vartheta + \frac{1}{2}a^2\sin^2\vartheta\right) }{\sin^2\vartheta},
	\]
	and so the function $\widehat{R}$ can be now written as:
	
	\begin{align}
	\widehat{R} =& \frac{L_z^2 \left( -\frac{a^2}{2} + 2r - r^2 - \frac{1}{2}a^2\cos^2\vartheta + \frac{1}{2}a^2\sin^2\vartheta\right) }{\sin^2\vartheta} - \nonumber\\
	&- 4 a r E L_z - \mu^2 \rho^2 \Delta + \Sigma^2 E^2 - Q \Delta + \nonumber \\ 
	&+ \Delta\cos^2\vartheta\left(\frac{L_z^2}{\sin^2\vartheta} + a^2\left(\mu^2 - E^2\right)\right).
	\end{align}
	
	Substituting $\Sigma$ and $\Delta$ by their values, defined on equations \ref{eq:termdef}, we obtain
	\begin{align}
	\widehat{R} =\,& \frac{a^2L_z^2}{2} - \Delta Q - 4aL_zrE + a^4E^2 + 2a^2r^2E^2 + r^4E^2 - a^2r^2\mu^2 + \nonumber \\
	&+ 2r^3\mu^2 - r^4\mu^2 + \nonumber\\
	&+ cos^2\vartheta\left(-a^4E^2 + 2a^2rE^2 - a^2r^2E^2 \right) + \nonumber \\
	&+ \sin^2\vartheta\left( -a^4E^2 + 2a^2rE^2 - a^2r^2E^2 \right) + \nonumber\\
	&+ \cot^2\vartheta\left( \frac{a^2L_z^2}{2} - 2L_z^2r + L_z^2r^2 \right) + \nonumber \\
	&+ \csc^2\vartheta\left( -\frac{a^2L_z^2}{2} + 2L_z^2r - L_z^2r \right).
	\end{align}
	
	We can simplify the last two pairs of terms using that $\sin^2 \vartheta + \cos^2 \vartheta = 1$ and that $\csc^2\vartheta - \cot^2\vartheta = 1$:
	\begin{align}
	\widehat{R} =\,& \frac{a^2L_z^2}{2} - \Delta Q - 4aL_zrE + a^4E^2 + 2a^2r^2E^2 + r^4E^2 - a^2r^2\mu^2 + \nonumber \\
	&+ 2r^3\mu^2 - r^4\mu^2 + \left(-a^4E^2 + 2a^2rE^2 - a^2r^2E^2 \right) + \nonumber \\
	&+ \left( -\frac{a^2L_z^2}{2} + 2L_z^2r - L_z^2r \right).
	\end{align}
	
	Factoring out common terms in the last two addends, using the definition of $\Delta$ and \autoref{eq:defP}, we simplify a little bit more:
	\begin{align}
	\widehat{R} =\,& \frac{a^2L_z^2}{2} - \Delta Q - 4aL_zrE + a^4E^2 + 2a^2r^2E^2 + r^4E^2 - a^2r^2\mu^2 + \nonumber \\
	&+ 2r^3\mu^2 - r^4\mu^2 + a^2E^2\left(-a^2 + 2r - r^2 \right) + L_z^2\left( -\frac{a^2}{2} + 2r - r^2 \right) = \nonumber \\
	=\,& \frac{a^2L_z^2}{2} - \Delta Q - 4aL_zrE + a^4E^2 + 2a^2r^2E^2 + r^4E^2 - a^2r^2\mu^2 + \nonumber \\
	&+ 2r^3\mu^2 - r^4\mu^2 - a^2E^2\Delta + L_z^2\Delta + \frac{a^2L_z^2}{2} = \nonumber \\
	=\,& \overbrace{a^2L_z^2 + a^4E^2 + 2a^2E^2r^2 + r^4E^2 - 2L_zr^2Ea - 2a^3L_zE}^{P^2} + \nonumber\\
	&+ 2L_zr^2Ea + 2a^3L_zE - \Delta Q - 4aL_zrE - a^2E^2\Delta - L_z^2\Delta - \nonumber \\
	&- a^2r^2\mu^2 + 2r^3\mu^2 - r^4\mu^2 - \Delta Q = \nonumber \\
	=\,& P^2 + \mu^2\overbrace{\left( -r^4 + 2r^3 - a^2r^2 \right)}^{\Delta r^2} - L_z^2\Delta - (a^2E^2)\Delta + 2L_zEar^2 + \nonumber\\
	&+ 2a^2L_zE - 4aL_zrE - Q\Delta = \nonumber\\
	=\,& P^2 + r^2\mu^2\Delta + L_z^2\Delta - a^2E^\Delta - 2L_zaE\left(2r - r^2 - a^2\right) - Q\Delta = \nonumber\\
	=\,& P^2 + r^2\mu^2\Delta - L_z^2 \Delta - a^2E^2\Delta - 2L_zaE\Delta - Q\Delta = \nonumber \\
	=\,& P^2 - \Delta \left( r^2\mu^2 + Q + \left(L_z - aE \right)^2 \right)
	\end{align}
	
	This proves that $\widehat{R} = R$, and therefore proves \autoref{eq:req}.
	
	\subsection*{$\dot{\vartheta}$ Equation}
	
	Using the definition of the Carter constant (\autoref{eq:carter}), we can directly find the value of $p_\vartheta$
	\begin{align}
	p_\vartheta^2 &= Q + a^2E^2\cos^2\vartheta - a^2\mu^2\cos^2\vartheta - \frac{L_z^2}{\sin^2\vartheta}\cos^2\vartheta = \\
	&= Q - \cos^2\vartheta\left( a^2\left(\mu^2 - E^2 \right) `\frac{L_z^2}{\sin^2\vartheta} \right).
	\end{align}
	
	Therefore, \autoref{eq:inittheta} is directly equivalent to the following one:
	\begin{equation}
	\rho^2 \dot{\vartheta} = \sqrt{\Theta},
	\end{equation}
	where $\Theta$ is defined in \autoref{eq:defTheta}. This finally proves \autoref{eq:thetaeq}.
\end{proof}

	
\section{Variational Formulation}
\label{sec:variational}

The study of the variational characterization of geodesics gave us an interesting result, from which we can change the problem of finding a geodesic by an equivalent one: to find the solution of a variational problem.

\autoref{pro:variationalgeodesic} states that $\gamma$ is a geodesic if and only if
\[
\frac{dE_f}{ds}(0) = 0.
\]

This lead us to understand geodesics ---we could even define them that way--- as the critical points of the energy, and yields a variational problem whose Lagrangian depends on the proper variation $f(s,t)$.

Using the characterization from \autoref{def:energy}, the variational problem equivalent to finding the equations of motion for $\gamma$ has the following Lagrangian:
\begin{equation}
	\label{eq:1stlagrangian}
	\mathcal{L} = g\left( \frac{df}{dt}, \frac{df}{dt} \right)
\end{equation}

This Lagrangian can now be expressed in terms of $\gamma$ components. We can then write the usual form of the Lagrangian:
\begin{align}
\mathcal{L} =\,& \frac{1}{2} \mathbf{v}^\mu \mathbf{v}_\mu = \\
=\,& \frac{1}{2} \Biggl( \dot{t}\biggl( -\dot{\varphi}^2\omega\varpi^2 + \dot{t} \left( -\alpha^2 + \omega^2\varpi^2 \right) \biggr) +\\
&\quad + \dot{r}^2\frac{\rho^2}{\Delta} + \dot{\vartheta}\rho^2 + \dot{\varphi}\left( \dot{\varphi} r^2 - \dot{t} \omega \varpi^2 \right) \Biggr)
\end{align}

This is similar to the formalism we developed in the previous section, but we can now consider the Hamiltonian version of this formulation, which appear simply by applying the Legendre transform:
\begin{equation*}
	\mathcal{H} = \sum p_i q_i - \mathcal{L}.
\end{equation*}

Our goal now is to recover the system described at \cite[Eq. (A.15)]{thorne15}. In order to do that, we can rewrite $\mathcal{H}$ as follows:
\begin{equation}
	\label{eq:hamiltonian}
	\mathcal{H} = \frac{p_r^2 \Delta}{2\rho^2} + \frac{p_\vartheta^2}{2\rho^2} + \mathfrak{f},
\end{equation}
where $\mathfrak{f}$ is the function consisting on the remaining terms of $\mathcal{H}$.

Although $\mathfrak{f}$ is completely defined and can be written as is, we are using the fact that $\mathcal{H} = \frac{-\mu^2}{2}$, and write it using the remaining terms.

Let us first work a little bit more on $\mathcal{H}$, rewriting \autoref{eq:hamiltonian}. First of all, we realize that we can write the definition of the components of $\mathbf{p}_\alpha$ from \autoref{eq:lowerv}:
\begin{align}
	\label{eq:pteq}
	p_t &= -\dot{t}\alpha^2 - \dot{\varphi}\omega\varpi^2 + \dot{t}\omega^2\varpi^2 \\
	\label{eq:preq}
	p_r &= \frac{\dot{r}\rho^2}{\Delta}\\
	\label{eq:pthetaeq}
	p_\vartheta &= \dot{\vartheta}\rho^2\\
	\label{eq:pphieq}
	p_\varphi &= \dot{\varphi}\varpi^2 - \dot{t}\omega\varpi^2\\
\end{align}

Now, using these equations and the ones obtained in \autoref{theo:firsteqs}, we can rewrite $\mathcal{H}$ as follows:
\begin{align*}
	\mathcal{H} &= \frac{p_r^2 \Delta}{2\rho^2} + \frac{p_\vartheta^2}{2\rho^2} + \mathfrak{f} \eqnote{=}{\ref{eq:preq}, \ref{eq:pthetaeq}} \frac{\left( \frac{\dot{r}\rho^2}{\Delta} \right)^2 \Delta}{2\rho^2} + \frac{\left(\dot{\vartheta}\rho^2\right)^2}{2\rho^2} + \mathfrak{f} = \\
	&= \dot{r}^2\frac{\rho^2}{2\Delta} + \dot{\vartheta}^2\rho^2 + \mathfrak{f} \eqnote{=}{\ref{eq:req}, \ref{eq:thetaeq}} \frac{R}{\rho^4 }\frac{\rho^2}{2\Delta} + \frac{\Theta}{\rho^4}\rho^2 + \mathfrak{f} =\\
	&= \frac{R}{2\rho^2\Delta} + \frac{\Theta}{\rho^2} + \mathfrak{f}.
\end{align*}

We can use now an interesting property of the Hamiltonian, which is a conserved quantity itself.

We know that $\mathcal{H} = \frac{1}{2} \mathbf{v}^\alpha \mathbf{v}_\alpha$. Taking into account that $\mathbf{v}$ is a normalized timelike vector, we have
\[
	\vert \mathbf{v} \vert = - \frac{\mu}{2}.
\]

Then, it is clear that
\[
	\mathcal{H} = \frac{-\mu^2}{2}
\]
and we can finally express $\mathfrak{f}$ easily:
\[
	\mathfrak{f} = - \frac{R + \Delta \Theta}{2\Delta\rho^2} - \frac{\mu^2}{2}.
\]

The final version of the Hamiltonian is:
\begin{equation}
	\mathcal{H} = \frac{p_r^2 \Delta}{2\rho^2} + \frac{p_\vartheta^2}{2\rho^2} - \frac{R + \Delta \Theta}{2\Delta\rho^2} - \frac{\mu^2}{2}.
\end{equation}

From the general Hamilton's equations:
\[
	\dot{p}_i = -\pd{\mathcal{H}}{q_i}, \quad \dot{q}_i = \pd{\mathcal{H}}{p_i},
\]
which in this case read as follows
\begin{align*}
	\dot{r} &= \pd{\mathcal{H}}{p_r}, \quad \dot{\vartheta} = \pd{\mathcal{H}}{p_\vartheta}, \quad \dot{\varphi} = \pd{\mathcal{H}}{p_\varphi} \\
	\dot{p}_r &= - \pd{\mathcal{H}}{r}, \quad \dot{p}_\vartheta = - \pd{\mathcal{H}}{\vartheta},
\end{align*}
we obtain the expected first order numerically stable system:
\begin{align}
	\dot{r} &= \frac{\Delta}{\rho^2} p_r \\
	\dot{\vartheta} &= \frac{1}{\rho^2}p_\vartheta \\
	\dot{\varphi} &= \pd{}{p_\varphi}\left( \frac{R + \Delta\Theta}{2\Delta\rho^2} \right) \\
	\dot{p}_r &= - \pd{}{r} \left( - \frac{\Delta}{2\rho^2}p_r^2 - \frac{1}{2\rho^2}p_\vartheta^2 + \left( \frac{R + \Delta\Theta}{2\Delta\rho^2} \right) \right) \\
	\dot{p}_\vartheta &= - \pd{}{\vartheta} \left( - \frac{\Delta}{2\rho^2}p_r^2 - \frac{1}{2\rho^2}p_\vartheta^2 + \left( \frac{R + \Delta\Theta}{2\Delta\rho^2} \right) \right).
\end{align}