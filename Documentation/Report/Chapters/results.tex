\chapter{Results}

The implemented software fulfils the objectives and requirements set at the design stage. It can ray trace arbitrary geodesics from the point of view of a camera, arbitrarily placed on a Kerr spacetime, allowing the user to plot the simulated geodesics both as photographies and as three dimensional projections.

Furthermore, the implemented \ac{ODE} solver accuracy was successfully tested, while the speed up of the \ac{GPU} parallelized code was proved to be very high.

\begin{figure}[bth]
	\myfloatalign
	\includegraphics[width=\linewidth]{gfx/bh_texture_disk}
	\caption[Cinematographic textured image]{Cinematographic textured image}
	\label{fig:blackhole}
\end{figure}

\section{Ray Tracer Features}

The ray tracer generates images with a wide range of possibilities. First of all, it can render images that simulate photographies made with a common camera near the black hole. An accretion disk can be added to the black hole, showing the curvature of the light near its surroundings. For both the disk and the celestial sphere, arbitrary textures can be added, letting the user experiment with the light and the distortions produced by the greatly curved spacetime near the Kerr black hole.

\autoref{fig:blackhole} is one example of a fully featured image with a cinematographic look: the celestial sphere is textured with an image of the Milky Way; an accretion disk around the black hole is added and a texture on the disk is rendered.

\begin{figure}[bth]
	\myfloatalign
	\includegraphics[width=\linewidth]{gfx/bh_texture_nodisk}
	\caption[Textured image without accretion disk]{Textured image without accretion disk}
	\label{fig:blackholenodisk}
\end{figure}

\autoref{fig:blackholenodisk} shows the same previous image, but rendered without the disk to better see the distortion produced by the black hole.

Furthermore, the rendered images can plotted as three dimensional projections, letting the user observe the paths followed by the geodesics in order to understand how they were curved.

An example of this feature can be seen on \autoref{fig:3dprojection}, where four snapshots of the three dimensional projection are shown. Blue lines represent geodesics that come from the celestial sphere, red lines are geodesics that never existed, as they would have originated inside the black hole's horizon; finally, the green lines are the geodesics whose origin is on the accretion disk.

The computed information can rendered independently as a three dimensional projection or as an image. \autoref{fig:3dprojectionimage} shows the photography whose three dimensional representation was depicted on \autoref{fig:3dprojection}. In this case, the disk is textured with a coloured patched texture.

\begin{figure}[bth]
	\myfloatalign
	\subfloat[Top view.]
	{\frame{\includegraphics[width=.45\linewidth]{gfx/3d_01_top}}} \quad
	\subfloat[Right view.]
	{\frame{\includegraphics[width=.45\linewidth]{gfx/3d_01_right}}} \\
	\subfloat[Perspective view from behind.]
	{\frame{\includegraphics[width=.45\linewidth]{gfx/3d_01_perspective1}}} \quad
	\subfloat[Perspective view from the front side.]
	{\frame{\includegraphics[width=.45\linewidth]{gfx/3d_01_perspective2}}}
	\caption[Three dimensional representation of the geodesics]{Three dimensional representation of the geodesics}\label{fig:3dprojection}
\end{figure}

\begin{figure}[bth]
	\myfloatalign
	\includegraphics[width=\linewidth]{gfx/3d_01_image}
	\caption[Photography generated by the scenario rendered in \autoref{fig:3dprojection}]{Photography of the three dimensional scenario rendered in \autoref{fig:3dprojection}}
	\label{fig:3dprojectionimage}
\end{figure}

\section{Computational results}


\subsection{Runge-Kutta Solver Accuracy}

The \ac{RK} solver has been tested not only against the geodesics \ac{ODE} system, but against usual functions whose analytic expression is known. This basic test was done with two purposes:
\begin{enumerate}
	\item To test that the solver was accurate.
	\item To study the behaviour of the automatic step size computation.
\end{enumerate}

One example of this test can be seen on \autoref{fig:stepsize}, that shows the behaviour of the \ac{RK} solver on the Airy function $Bi(x)$. The orange line is its analytics expression, while the blue points are the solution computed by the \ac{RK} solver of the \ac{ODE}
\[
	\frac{d^2y}{dx^2} - xy = 0.
\]

\begin{figure}[bth]
	\myfloatalign
	\includegraphics[width=\linewidth]{gfx/analytic}
	\caption[\ac{RK} solver in an analytic function]{\ac{RK} solver in an analytic function}
	\label{fig:stepsize}
\end{figure}

It can be seen how the automatic step computation algorithm works smoothly: when the function can be approximated as a straight line, the step, which is the space between successive points, is very large. However, in intervals where the function changes rapidly its curvature, the algorithm reduces the step in order to better approximate the function value.

This behaviour can also be seen on the geodesics \ac{ODE} system. \autoref{fig:kretschmann} shows the normalized values of four quantites:
\begin{enumerate}
	\item The Kretschmann invariant, which measures the curvature of the spacetime, and which is known to diverge in the singularity.
	\item The distance to the black hole.
	\item The value of the $\vartheta$ angle.
	\item The step computed by the automatic step algorithm.
\end{enumerate}

When the lightlike particle approaches the black hole; \ie, when the distance to the black hole decreases and, as a consequence, the Kertschmann increases, the system becomes unstable, and so the algorithm reduces the step size in order to better approximate its value.

In fact, both the step line and the radius line are very similar, showing the correlation between the computed step and the distance to the black hole centre.

\begin{figure}[bth]
\myfloatalign
\includegraphics[width=\linewidth]{gfx/kretschmann}
\caption[Step, $r$, $\vartheta$ and Kretschmann]{Step, radius, $\vartheta$ and Kretschmann}
\label{fig:kretschmann}
\end{figure}

\subsection{Efficiency}

Regarding the efficiency of the ray tracer, a benchmark against a \ac{CPU} implementation has been done.

The \ac{CPU} implementation has essentially the same code that the \ac{GPU}-parallelized version, except for the obvious changes that were made to adapt the code to the \ac{CUDA} grid.

\autoref{fig:speedup} shows two benchmarks using both versions of the ray tracer: one with a Kerr spacetime where the black hole has a spin of $a = 0.0001$ and another one where the spin is $a = 0.999$.

For both benchmarks, the speed up is plotted, \ie, the line represented for each of them shows how many times faster is the \ac{GPU}-parallelized version against the \ac{CPU} implementation.

\begin{figure}[bth]
	\myfloatalign
	\includegraphics[width=.8\linewidth]{gfx/speedup}
	\caption[Speed up with different spins]{Speed up with different spins}
	\label{fig:speedup}
\end{figure}


\section{Scientific results}

\subsection{Spin}

The effect of the black hole spin on its surroundings has been studied in two ways.

First of all, we have generated images without any accretion disk and without textures for the celestial sphere. This gives us binaries images where the black pixels represent the shadow of the black hole and the white pixels the geodesics that came from the celestial sphere.

\autoref{fig:shadow} shows four images taken with the same camera, which is placed on the equatorial plane, with a null speed and focusing the black hole's centre. The only change between images is the black hole's spin.

The first one shows a perfect sphere. This is the edge case where the Kerr metric can be reduced to the more simple Schwarzschild metric.

\autoref{fig:shadow-b} and \autoref{fig:shadow-c} shows the same image with an increased spin. There is a slight change on the virtual position of the shadow and, although it is difficult to see, the shape is not circular.

The edge case, depicted on \autoref{fig:shadow-d}, makes clear the effect of a large spin on the shadow of the black hole. As it curves the geodesics rapidly, it is seen slightly moved to the right and with a flat side on the left: if the spin were of this same magnitude but negative, the image would be vertically mirrored.

\begin{figure}[bth]
	\myfloatalign
	\subfloat[Spin $\approx$ 0]
	{\frame{\includegraphics[width=.45\linewidth]{gfx/bh_shadow_spin0001}}} \quad
	\subfloat[Spin = 0.25]
	{\label{fig:shadow-b}%
		\frame{\includegraphics[width=.45\linewidth]{gfx/bh_shadow_spin25}}} \\
	\subfloat[Spin = 0.75]
	{\label{fig:shadow-c}%
		\frame{\includegraphics[width=.45\linewidth]{gfx/bh_shadow_spin75}}} \quad
	\subfloat[Spin $\approx$ 1]
	{\label{fig:shadow-d}%
		\frame{\includegraphics[width=.45\linewidth]{gfx/bh_shadow_spin999}}}
	\caption[Black hole shadow for different spins]{Black hole shadow for different spins}\label{fig:shadow}
\end{figure}

It is also interesting to see what happens with straight lines that fall inside the black hole.

Imagine an accretion disk around the black hole, with its inner radius minor than the horizon radius and an infinite outer radius. If we visualize this disk with a patched texture made by white and red squares, we can see the effect of the spin on its shape.

\autoref{fig:xmas-a} shows the most simple version of this scenario, where the black hole does not rotate: straight lines falling inside the black hole on the equatorial plane remain intact.

Taking this as the base case, we can see what happens when we increase the black hole's spin. \autoref{fig:xmas-b} and \autoref{fig:xmas-c} shows the scenario for spins of $0.25$ and $0.75$. The straight lines start to rotate accordingly with the black hole.

The most interesting image is depicted on \autoref{fig:xmas-d}, where we see an extreme spin of nearly 1. The lines curve greatly when they approach the shadow, and start rotating rapidly when they are close to the horizon.

\begin{figure}[bth]
	\myfloatalign
	\subfloat[Spin $\approx$ 0]
	{\label{fig:xmas-a}%
		\frame{\includegraphics[width=.45\linewidth]{gfx/bh_xmas_spin0001}}} \quad
	\subfloat[Spin = 0.25]
	{\label{fig:xmas-b}%
		\frame{\includegraphics[width=.45\linewidth]{gfx/bh_xmas_spin25}}} \\
	\subfloat[Spin = 0.75]
	{\label{fig:xmas-c}%
		\frame{\includegraphics[width=.45\linewidth]{gfx/bh_xmas_spin75}}} \quad
	\subfloat[Spin $\approx$ 1]
	{\label{fig:xmas-d}%
		\frame{\includegraphics[width=.45\linewidth]{gfx/bh_xmas_spin999}}}
	\caption[Black hole shadow for different spins]{Black hole shadow for different spins}\label{fig:xmas}
\end{figure}	