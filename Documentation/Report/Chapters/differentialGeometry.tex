\chapter{Introduction to differential geometry}

\section{Differentiable manifolds}

Roughly speaking, a manifold is a topological space that, locally, looks like the Euclidean space $\R^n$. This similitude is essential, and will let us control the manifold as if we were working in the Euclidean space; generally, the definitions concerning manifolds and the properties proved from them will be based on the known properties of $\R^n$.

The following definition specifies the formal concept of a topological manifold:

\begin{definition}[N-dimensional topological manifold]
    Let $M^n$ be an $n$-dimensional topological space. The space $M^n$ is called a topological manifold if the following properties are satisfied:
    \begin{enumerate}
        \item $M^n$ is locally homeomorphic to $\R^n$. \label{def:manifold:homeo}
        \item $M^n$ is a Hausdorff space. \label{def:manifold:haussdorf}
        \item $M^n$ has a countable topological basis. \label{def:manifold:basis}
    \end{enumerate}
\end{definition}

The first property states that, for every point $p \in M^n$, there exists an open neighbourhood $U \subset M^n$ of $p$ and a homeomorphism
\[
    h \colon U \to V
\]

with $V \subset \R^n$ an open set.

One could think that the Hausdorff property is redundant, as the local homeomorphism may imply this topological characteristic. This is not true, and the usual counterexample is the line with two origins.

Let $M = \R \cup p$ be the union of the real line and a point $p \notin \R$. Define a topology in this space with $\R \subset M$ as an open set and the neighbourhoods of $p$ being the sets $(U \setminus \{0\}) \cup \{p\}$, where $U$ is a neighbourhood of $0 \in \R$. This space is locally Euclidean but not Hausdorff: the intersection of any two neighbourhoods of the points $0 \in \R$ and $p$ is non-empty.

\begin{figure}[bth]
    \myfloatalign
    \begin{tikzpicture}
      \draw[thick] (-5,0) -- (-0.05,0);
      \draw[very thick,<->] (-1,0) -- (-0.05,0);
      \draw[fill] (0,0) circle [radius=0.05];
      \node[below] at (0,0) {0};
      \draw[thick] (0.05,0) -- (5,0);
      \draw[very thick,<->] (0.05,0) -- (1,0);
      \node[right] at (5,0) {$\R$};

      \draw[fill,Maroon] (0,0.5) circle [radius=0.05];
      \node[right,Maroon] at (0,0.5) {$p$};
    \end{tikzpicture}
    \caption[Line with two origins]{Line with two origins.}
    \label{fig:2origin}
\end{figure}

The last property of the definition will be proven key in our study, as it will let us define metrics on the manifold.

\subsection{Charts}

The main characteristic of the manifolds, its ressemblance\unsure{Is this word ok?} to the Euclidean space, have to be exploited in order to understand the nature of the mathematical object.

The conceptual space where the manifolds live can be thought as the Plato's world of Ideas\unsure{Maybe too much}, where everything is pure but cannot be understood without studying particular examples.

The idea of the manifold will be understood, then, taking pieces of the manifold and lowering\unsure{Mmm... not sure about the word.} them to the real word; \ie, the Euclidean space, where we will be able to \emph{physically} touch the manifold.

The essential tool to make this happen will be the coordinate charts. These tools are like prisms to see the manifold from the Euclidean perspective, and they will let us grasp the nature of the ideal concept of a manifold.

\begin{definition}[Coordinate chart]
    A \emph{coordinate chart} ---or \emph{coordinate system}--- in a topological manifold $M^n$ is a homeomorphism $h \colon U \to V$ from an open subset of the manifold $U \subset M$ onto an open subset of the Euclidean space $V \subset \R^n$.

    We call $U$ a \emph{coordinate neighbourhood} in $M$.
\end{definition}

One single chart may not cover the whole manifold. In order to completely understand it, we need a set of charts that describe it completely.

\begin{definition}[Coordinate atlas]
    Let
    \[
    A = \{h_\alpha \colon U_\alpha \to V_\alpha / \alpha \in I\}
    \]
    be a set of coordinate charts in a topological manifold $M^n$, where $I$ is a family of indices and the open subsets $U_\alpha \subset M$ are the corresponding coordinate neighbourhoods.

    $A$ is said to be an \emph{atlas} of M if every point is covered with a coordinate neighbourhood; \ie, if $\cup_{\alpha \in I} U_\alpha = M$.
\end{definition}

% Examples?

\subsection{Differentiable structures}

The concept of manifold is quite general and includes a vast set of examples. We can impose, however, some properties on the smoothness of the manifold to restrict the objects we will work with.

This section introduces the concept of differentiable structure, whose definition is key in the later description of differentiable manifolds, the core concept of this chapter.

The first question in this study is the following: a chart describe perfectly a single piece of the manifold, but what happens when the domains of a pair of charts overlap? The following two definitions specify the concepts involved in this question.

% TODO: Add the usual conmutative diagram for the transition maps

\begin{definition}[Transition map]
    Let $M^n$ be a manifold and $(U, \phi)$, $(V, \psi)$ a pair of coordinate charts in $M^n$ with overlapping domains, that is:
    \[
        U \cap V \neq \emptyset
    \]

    The homeomorphism between the open sets of the Euclidean space $\R^n$,
    \[
        \psi \circ \phi^{-1} \colon \phi(U \cap V) \to \psi(U \cap V),
    \]
    is called a \emph{transition map}.
\end{definition}

\begin{definition}[Smooth overlap]
    Two charts $(U, \phi)$, $(V, \psi)$ are said to overlap smoothly if their domains are disjoint ---\ie, if $U \cap V  = \emptyset$--- or if the transition map $\psi \circ \phi^{-1}$ is a diffeomorphism.
\end{definition}

The description of two charts that overlap smoothly can be naturally extended to the concept of smooth atlas, that will make possible to do calculus on the manifold.

\begin{definition}[Smooth coordinate atlas]
    An atlas $A$ is said to be smooth if every pair of charts in $A$ overlap smoothly.
\end{definition}

But what happens if we define two different atlases in the manifold? Will the calculus depend on this choice? Fortunately we can find, for each manifold, one particular atlas that contain every other atlas defined there. It is formally described in the following definition and its uniqueness is proved in \autoref{prop:max-atlas-uniq}.

\begin{definition}[Complete atlas]
    A \emph{complete atlas} ---or \emph{maximal atlas}--- on $M^n$ is a smooth atlas that contains each coordinate chart in $M^n$ that overlaps smoothly with every coordinate chart in $M^n$.
\end{definition}

\begin{proposition}[Complete atlas uniqueness]
    Let $M^n$ be a topological manifold.

    \begin{itemize}
        \item Every smooth atlas on $M^n$ is contained in a complete atlas.
        \item Two smooth atlas on $M^n$ determine the same complete atlas if and only if its union is a smooth atlas.
    \end{itemize}
    \label{prop:max-atlas-uniq}
\end{proposition}


\begin{proof}
 	Let $A$ be a smooth atlas on $M^n$ and define $A'$ as the set of all $n$-dimensional coordinate charts that overlaps smoothly with every chart on $A$. We are going to see that $A'$ is a complete atlas.
 	
 	It is trivial to see that $A'$ is an atlas, since $A \subset A'$ and $A$ is an atlas. The smoothness of the atlas is a consequence of the fact that smoothness is a local property \fixme{Finish.}
\end{proof}

\begin{definition}[Differentiable manifold]
	A \emph{differentiable manifold} is a pair $(M, A)$, where $M$ is a topological manifold and $A$ is a complete atlas.
\end{definition}

\begin{example}
	The concept of differentiable manifold is, probably, the most important idea throughout all this work. Let us see then some examples in order to better understand that these spaces we will going to work with are not that abstract ---although they can be---.
	
	\begin{enumerate}
		\item The Euclidean space $\R^n$ is a differentiable manifold considering the identity map as its atlas.
		\item Every \emph{smooth surface}\footnote{We consider the definition of smooth surface seen in a basic course of curves and surfaces: a subset of $\R^3$ such that every point is covered by the image of a differentiable map whose restriction to an open subset containing the point is an homeomorphism and whose differential is a monomorphism.} of $\R^3$ is an example of a differentiable manifold. As a subset of $\R^3$, the local homeomorphism, the Hausdorff property and the countable basis are trivial. Furthermore, the definition of smooth surface gives us for free the complete atlas.
		\item The sphere $S^n$ is an $n$-dimensional differentiable manifold. As an atlas we can consider the union of the two stereographic projections onto $\R^n$ from the north and south poles.
	\end{enumerate}
\end{example}

\subsection{Differentiable maps on manifolds}

The concept of differentiable maps on manifolds is the first one in which we are going to generalize concepts from the Euclidean space using the local homeomorphism.

The idea is simple: we know how to build differentiable maps between open sets of $\R^n$, so we are going to define differentiability between manifolds going through the images of the coordinate neighbourhoods of the points.

As the differentiability is a local concept, being the manifolds locally Euclidean is enough to generalize it.

\begin{definition}
	Let $F \colon M \to N$ be a map between two differentiable manifolds: $M$ and $N$. F is said to be \emph{differentiable} or \emph{smooth} if the following conditions are satisfied:
	\begin{enumerate}
		\item There is a chart $(U, \varphi)$ for every point $p \in M$ and another one, $(V, \psi)$ for its image, $F(p) \in N$, such that $p \in U$, $F(p) \in V$ and $F(U) \subset V$.
		\item The map $\psi \circ F \circ \varphi^{-1} : \varphi(U) \to \psi(V)$ is differentiable in the usual sense.
	\end{enumerate}
\end{definition}

This definition includes also the case in which $M$, $N$ or even both of them are the euclidean spaces $\R^m$ and $\R^n$. There is no ambiguity between this and the euclidean definition of smoothness, as one can take the identity map as coordinate chart when one of the manifolds is an euclidean space and the usual definition will be found.

From this definition it is trivial to prove that, if a family of smooth maps cover a manifold with the maps being equal where their images overlap, a unique smooth function that is equal to each individual map on its image can be built.

Furthermore, it is easy to see that the identity of a manifold, the coordinate charts and the composition of smooth functions are smooth. Smoothness also implies continuity.

As well as the definition of smoothness, the definition of diffeomorphism can be generalized to manifolds, being \autoref{def:diffeo} its formal expression.

\begin{definition}[Diffeomorphism]
	\label{def:diffeo}
	A function $f \colon M \to N$ between two manifolds is said to be a \emph{diffeomorphism} if it is a smooth bijective map with its inverse being also smooth.
	
	When there exists such map, $M$ and $N$ are said to be diffeomorphic.
\end{definition}

\subsection{Tangent space}

Once we know what a differentiable function is, the next step we need to take in order to set up a proper place to do calculus is to define its differential.

first, let us remember some concepts about regular surfaces on $\R^3$. Let $S, S'$ be two regular surfaces on $\R^3$ and let $f \colon S \to S'$ be a differentiable map between them. The differential of $f$ on $p \in S$ was defined as a function that transforms tangent vectors to the first surface into tangent vectors to the second,
\[
	(df)_p \colon T_p S \to T_{f(p)} S'.
\]

What can we learn from this? Our goal is to define the differential of a differentiable map between \emph{manifolds}. It would be ideal that it generalizes the notion we already have about differentials on surfaces, so it is mandatory to first generalize the concept of tangent plane.

The tangent plane to a regular surface on one of its points $p$ is, as we know, the vector subspace of all the tangent vectors to the point. This vector space was shown to be isomorphic to the space of directional derivatives on $p$. Instead of trying to generalize the concept of tangent vector, the idea we will follow is to extend the notion of directional derivatives, building the new \emph{tangent plane}-like space from these.

Basically, a directional derivative is a linear map that satisfies the Leibniz rule, so we are going to define a tangent vector to a manifold as such.

From now on, $\mathcal{F}(M)$ will note the set of all the smooth real-valued functions on a manifold $M$:
\[
	\mathcal{F}(M) \defeq \{f \colon M \to \R / \textrm{f is smooth} \}
\]

\begin{definition}[Tangent vector]
	Let $M$ be a manifold and $p \in M$ one of its points. A \emph{tangent vector to M on p} is a linear application that maps smooth functions on $M$ to real values and that satisfies the Leibniz rule; \ie, a function
	\[
		v \colon \mathcal{F}(M) \to \R
	\]
	that
	\begin{enumerate}
		\item is linear: $v(af + bg)\bigr|_p = av(f)\bigr|_p + bv(g)\bigr|_p$ and
		\item satisfies the Leibniz rule: $v(fg) = v(f)\bigr|_p g(p) + f(p)v(g)\bigr|_p$.
	\end{enumerate}
\end{definition}

It is trivial to see that the directional derivative is a tangent vector to the well-known manifold $\R^n$. Being this \emph{directional derivative --- vector} duality clear, it is now natural to arrive to \autoref{def:tangentspace}.

\begin{definition}[Tangent space]
	\label{def:tangentspace}
	Let $M$ be a manifold and $p \in M$ one of its points. The \emph{tangent space to $M$ at $p$}, noted as $T_p(M)$, is the set of all tangent vectors to $M$ on $p$.
\end{definition}

\begin{remark}
	$T_p(M)$ is a vector space with the usual definitions of function addition and product by a scalar.
\end{remark}