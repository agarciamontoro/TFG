\chapter{Objectives}

The initial objectives for this work were:
\begin{enumerate}
	\item To have a good understanding of differential geometry in general and semi-Riemannian geometry in particular.
	\item To have a basic understanding of general relativity.
	\item To be fluent on the design of numerical codes parallelized with \acp{GPU}.
	\item To implement an algorithm that solves the differential equations derived from the mathematical study.
	\item To study the obtained results.
\end{enumerate}

All objectives have been fulfilled:
\begin{enumerate}
	\item \autoref{chapter:diffgeom} proves that the needed concepts, results and tools related to differential geometry have been understood. \autoref{chapter:semiriemannian} contains all the work done on the semi-Riemannian geometry field, and also proves that a good understanding of it has been acquired.
	\item Chapters \ref{chapter:einstein}, \ref{chapter:kerr} and \ref{chapter:equations} contains all the learned aspects of general relativity theory. This introduction to the theory fulfil the initial objective.
	\item Chapters \ref{chapter:design} and \ref{chapter:implementation} describe the algorithm implemented, which fulfils all the initial requirements.
	\item \autoref{chapter:results}, which is the study of the results obtained from the acquired mathematical knowledge using the implemented software proves that the final objective was also successfully fulfilled. 
\end{enumerate}

To achieve the objectives related to the mathematical background was difficult due to the inherent complexity of these particular fields. The implementation of the software was tricky and time-consuming, but a lot of unknown aspects on \ac{GPU} architectures, maybe collateral to the main objectives of this work but as important as them, have been learned.