\chapter{Mathematics}
\label{chapter:mathematicss}

The part summarised in this chapter set up the basic background needed to understand what we called a spacetime, the main mathematical object in which we developed our work.

The contents here are greatly summarised, and one cannot understand the needed mathematical background by just reading these three pages. The reader should look up the original text in order to understand the complexity and depth of the studied results, and this part of the abridgment has to be read just as a slight walk over the main concepts studied in the work.

\section{Introduction to Differential Geometry}
\label{chapter:diffgeom}

\subsection{Differentiable Manifolds}

The work started studying manifolds that, roughly speaking, are topological spaces that, locally, looks like the Euclidean space $\R^n$.	The concept of differentiable manifold was found to be, probably, the most important idea throughout all the work, along with differentiable maps on manifolds.

We knew how to build differentiable maps between open sets of $\R^n$, so we defined differentiability between manifolds going through the images of the coordinate neighbourhoods of the points. As the differentiability is a local concept, being the manifolds locally Euclidean was enough to generalize it.

But what about defining the differential of a differentiable map between \emph{manifolds}? It would be ideal that it generalized the notion we already had about differentials on surfaces, so it was mandatory to first generalize the concept of tangent plane.

The tangent plane to a regular surface on one of its points $p$ is, as we knew, the vector subspace of all the tangent vectors to the point. This vector space was shown to be isomorphic to the space of directional derivatives on $p$. Instead of trying to generalize the concept of tangent vector, the idea we followed was to extend the notion of directional derivatives, building the new \emph{tangent plane}-like space from these.

On the $\R^3$ surfaces scenario, it is not odd to define tangent vectors using their close relation with the curves on the surface. In order to obtain a better understanding of the manifolds tangent space, we defined what a curve on a manifold is and how a tangent vector on a point could be identified with them.

Finally, we proved every smooth curve was differentiable in the manifold sense, and having understood the duality between derivations and tangent vectors, we were able to naturally obtain the tangent vector to a curve on an instant $t_0\in I$.

\subsection{Affine Connections}
\label{sec:affineconnections}

Another important concept on manifolds was studied: the affine connections. They gave us in turn the tools to generalize the concept of directional derivative arriving to the definition of covariant derivative, which was shown to be a generalization of the directional derivative on $\R^n$. The concept of connection provided us with a way of derivating vectors along curves; \ie, we had then the possibility to consider the concept of \emph{acceleration} on curves on manifolds.


\section{Semi-Riemannian Geometry}
\label{chapter:semiriemannian}

This section introduced the notion of \emph{spacetime}, a mathematical object with a really interesting meaning in physics, as it models the geometry of the Universe. We needed first some basic concepts, such as the Lorentzian manifolds and the time orientation we were able to define on them. As these concepts are deeply studied in the original work, only the notion of spacetime will be detailed here:

A \emph{spacetime} is a four dimensional time oriented Lorentzian manifold. Roughly speaking, a time orientation is a map that assigns one of the two time cones defined on the tangent space to a point on a manifold; \ie, is a map that assigns a time orientatio in $T_pM$ for every $p \in M$.

\subsection{Semi-Riemannian Connections}

On \autoref{sec:affineconnections}, we studied connections on general manifolds, regardless of the addition of a metric. However, the existence of an interesting connection is naturally assured when one adds a semi-Riemannian metric to the manifold: the Levi-Civita connection, whose interest will be shown along this section.

We concludes then that, when restricted to the euclidean spaces, the covariant derivative and the directional derivative agree.

\subsection{Geodesics}

\begin{definition}[Geodesic]
	% doCarmo, 52
	Let $\gamma \colon I \to M$ be a curve on $M$. $\gamma$ is said to be a \emph{geodesic on $M$} when
	\[
	\frac{D}{dt}\left(\frac{d\gamma}{dt}\right) = 0 \quad \forall t \in I,
	\]
	that is, the vector field $\gamma'$ is parallel.
\end{definition}

It is a common abuse of notation, using that the covariant derivative is an actual derivation, to write that a curve $\gamma$ on $M$ is a geodesic when $\gamma'' = 0$.

One milestone on the development of this work was to obtain the equations that characterise geodesics: these kind of equations let us apply a numerical algorithm in order to obtain positions of particles moving on manifolds. As a spacetime can be modelled as a 4-dimensional manifold, its geodesics will tell us the path light follows.

As our primary objective was to study the movement of light, this first small step in understanding what a geodesic is and what equation it satisfies was very important.

\begin{remark}[Differential equations satisfied by a geodesic]
	
	Let $\gamma(t) = (x^1(t), \dots, x^n(t)$ be the coordinates of the curve on $U$. Using the expression of the covariant derivative and assuming $\gamma$ is a geodesic; \ie, that its velocity vector field is parallel, we can write:
	\[
	0 = \pd{}{x^k} \left( \frac{d^2x^k}{dt^2} + \Gamma^k_{ij} \frac{d x^i}{dt} \frac{d x^j}{dt} \right).
	\]
	
	We conclude that the \ac{ODE} system of order 2 given by
	\begin{equation}
	\label{eq:geodesic}
	\frac{d^2x^k}{dt^2} + \Gamma^k_{ij} \frac{d x^i}{dt} \frac{d x^j}{dt} = 0, \quad k = 1, \dots, n,
	\end{equation}
	describes a necessary condition for the curves on $M$ to be geodesics.
\end{remark}

\subsubsection*{Variational Characterization of Geodesics}

When working with geodesics, its natural, elegant definition, although really interesting for theoretical purposes, was not practical. This section of the work aimed to find a characterization that let us study geodesics and we found that geodesics are, in short, the critical points of the energy of every proper variation.

All details and results can be consulted in the original work.