\chapter{General Description}

The original work aimed to study how the light moves on the surroundings of black holes, developing a tool that uses this information to generate images of what an observer would see when getting closer to a body of this kind. This problem was well-known \cite{oneill83} \cite{oneill95} and had been widely studied.

As we do not usually have analytic expressions for the paths followed by photons, we were forced to work with numerical solutions, so it was found important to have a general-purpose, robust, stable and well documented code that could solve this problem, apart from the existing numerical implementations \cite{thorne15} \cite{chan13} that are either privative, are not well documented or are not adapted with a general purpose. The scientific community would take advantage of such a software, so the main goal of this work was to implement it.

The addressed problem was divided into two sub problems: to understand how the light moves near a Kerr black hole ---this yielded an \ac{ODE} system with no analytic solution--- and to implement a software ---a ray tracer--- that numerically solves the equations for the photons trajectories obtained with the previous study.

The original work was organised in three main parts:
\begin{enumerate}
	\item Mathematics: all the needed mathematical background was studied here, from basic concepts as the tensor algebra to more advanced topics as semi-Riemannian geometry.
	\item General Relativity: this part focused on the concept of spacetime and the equations for the photons trajectories near Kerr black holes.
	\item Computer Science: this part covered the computational solution to the problem of finding the path followed by light in the surroundings of a black hole. It described the design, studying \ac{GPGPU} techniques, describing the implemented solution and the results obtained. The implemented code was up to 125 times faster than non-parallelized solutions.
\end{enumerate}

These parts are summarised here in the three following chapters, that give an overall idea of the content detailed in the work, although proofs and details, as well as some basic definitions, will be omitted.

The source code of the software and its documentation, as well as the sources of this document, can be found on \href{https://github.com/agarciamontoro/TFG}{its public repository}\footnote{\url{https://github.com/agarciamontoro/TFG}}.