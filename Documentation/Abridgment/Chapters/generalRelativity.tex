\chapter{General Relativity}
\label{chapter:general-relativity}

The part summarised in this chapter introduced some notions to understand the main idea of general relativity theory; \ie, that gravity is indeed a consequence of the geometric nature of the spacetime and that the trajectories of particles near celestial bodies, such as black holes, that are not under any other force are characterised by the geodesics ---their null acceleration---.

\section{Geometric Description of Gravity}
\label{chapter:einstein}

This section aimed to obtain the Einstein field equations, that in detail the spacetime, its shape and its properties.

\subsection{Einstein Tensor of a Metric}

Before describing the Einstein field equations, we needed two basic concepts: the Einstein tensor of a metric and a stress-energy tensor.

Let $(M,g)$ be a spacetime. The symmetric 2-covariant tensor field $G \defeq \operatorname{Ric}-\frac{1}{2}Sg$, where Ric is the Ricci tensor of $g$ and $S$ the scalar curvature of $g$, is called the \emph{Einstein tensor of $g$}.

A \emph{stress-energy tensor} on $M$ is a symmetric 2-covariant tensor field $T$ such that $T(V,V) \geq 0$ for any timelike vector $V \in T_p M \; \forall p \in M$.

\subsection{Einstein Field Equation}

Consider now a stress-energy tensor $T$ on $(M,g)$ such that $T - \frac{1}{2}(\operatorname{trace}_g T) g$ is also a stress-energy tensor and that $T$ satisfies the conservation law $\operatorname{div} \widehat{T} = 0$, where $\widehat{T}$ is the 2-contravariant tensor field $g$-equivalent to $T$.

We said that the spacetime $(M,g)$ obeys the Einstein field equation with respect to $T$ (using the terminology of \cite[Sec. 6.2]{sachswu77}) if

\begin{equation}
\label{eq:einstein3}
T - \frac{1}{2}(\operatorname{trace}_g T)g = \Ric.
\end{equation}

Einstein field equation postulated how matter and radiation in a region of the universe can be described by a Lorentzian metric $g$.

If $(M,g)$ obeys the Einstein field equation with respect to $T = 0$, then, we proved that $Ric = 0$. Assuming then that $(M,g)$ obeys the Einstein field equation $Ric = 0$, then it is trivial that $T=0$.

That way, we were able to conclude that the \emph{vacuum Einstein field equation} was the mathematical way of expressing the absence of matter and radiation on the spacetime.



















\section{Kerr Spacetime}
\label{chapter:kerr}

The Kerr metric is an exact solution of the vacuum Einstein field equation that models a spacetime with an axially symmetric black hole rotating about an axis through its centre. The coordinate chart we used to describe each event occuring on the spacetime's manifold was defined by the \ac{BL} coordinates and the Kerr spacetime was parametrized by its angular momentum, $a \neq 0$; \ie, how fast its rotation is.

The Kerr metric is defined as follows:
\begin{align}
	\label{eq:kerrmetric}
	ds^2 = &-dt^2 + \rho^2\left(\frac{dr^2}{\Delta} + d\vartheta^2\right) + \left(r^2 + a^2\right)\sin^2\vartheta d\varphi^2 + \\
	\nonumber
	&+ \frac{2r}{\rho^2}\left(a\sin^2\vartheta d\varphi - dt\right),
\end{align}
where $\rho^2 = r^2 + a^2\cos^2\vartheta$ and $\Delta = r^2 + -2r + a^2$.

It was observed \cite[Sec. 2.1]{galindo14}, \cite[pp. 58-59]{oneill95} that the Kerr metric is stationary, axially symmetric and invariant when applying the transformation $t \to -t$, $\varphi \to -\varphi$, from where we saw that the energy of the particle and the angular momentum, both per unit mass and measured from the infinity, are conserved.

\subsection{Horizons and Singularities}

The Kerr metric is singular if $\rho^2 = 0$, which makes the Kretschmann invariant divergent, and that can be physically understood as a divergent curvature of the spacetime itself.

Assuming $\Delta = 0$, we obtained two coordinate singularities: $r_{\pm} \defeq M \pm \sqrt{M - a^2}$, which are called \emph{horizons}, as it was proved that, on them, the constant-$r$ hypersurfaces are null. The two horizons divide the spacetime in three regions:
\begin{enumerate}
	\item The region where $r > r_+$, that can be seen as the outside of the black hole.
	\item The region where $r_- < r < r_+$. Here, any particle falling through $r_+$ is forced to reach $r_-$, hence, the hypersurface defined by $r = r_+$ is called the \emph{event horizon}.
	\item The region where $r < r_-$, which contains the spacetime singularity.
\end{enumerate}











\section{Equations of Motion}
\label{chapter:equations}

In the work we aimed to find the most computationally stable equations of motion for $\gamma$, a free falling causal particle moving on a Kerr spacetime.

The classical equations derived from the definition of geodesic in terms of the Christoffel symbols, $\frac{d^2x^k}{dt^2} + \Gamma^k_{ij} \frac{d x^i}{dt} \frac{d x^j}{dt} = 0$, had some flaws that forced us to discard them: it is a second order system of differential equations, which does not really fitted the numerical algorithm to integrate them; furthermore, it was found very difficult to get the conserved quantities \emph{into} the equations.

However, the classical Hamiltonian formulation \cite[Sec. 33.5]{thorne73} gave us a first order system with the conserved quantities in the equations. In this formulation, however, another problem arose, as a pair of square roots appeared on the right hand side of the equations.

Finally, we were able to get rid of those square roots by defining the Hamiltonian and using a version of it that eased our analytical computations. This approach led us to a first order system with conserved quantities and which was numerically-friendly. After a long study detailed in the original work, the equations derived were the following:

\begin{theorem}[Equations of motion]
	\label{theo:eqsmotion}
	\begin{align}
	\dot{r} &= \frac{\Delta}{\rho^2} p_r \\
	\dot{\vartheta} &= \frac{1}{\rho^2}p_\vartheta \\ \label{eq:eqsmotionp}
	\dot{\varphi} &= \pd{}{p_\varphi}\left( \frac{R + \Delta\Theta}{2\Delta\rho^2} \right) \\ \label{eq:eqsmotionpr}
	\dot{p}_r &= - \pd{}{r} \left( - \frac{\Delta}{2\rho^2}p_r^2 - \frac{1}{2\rho^2}p_\vartheta^2 + \left( \frac{R + \Delta\Theta}{2\Delta\rho^2} \right) \right) \\ \label{eq:eqsmotionpt}
	\dot{p}_\vartheta &= - \pd{}{\vartheta} \left( - \frac{\Delta}{2\rho^2}p_r^2 - \frac{1}{2\rho^2}p_\vartheta^2 + \left( \frac{R + \Delta\Theta}{2\Delta\rho^2} \right) \right).
	\end{align}
\end{theorem}